\section{Ordini di grandezza}

\subsection{Misure adimensionali interazioni}

Potenziale tra 2 nucleoni $-Gm_H^2/r$ con $r=\frac{h}{2\pi m_Hc}$ (Pindet): in unit\'a $m_Hc^2$ $\alpha_G=\frac{2\pi Gm_H^2}{hc}\approx\num{e-39}$.

$\alpha_{SF}=\frac{e^2}{(4\pi\epsilon_0)\hbar c}$.

The ratio of three characteristic lengths:

the classical electron radius $r_e=(\frac{1}{4\pi\epsilon_0})\frac{e^2}{m_ec^2}$, the Bohr radius $a_0=(4\pi\epsilon_0)\frac{\hbar^2}{m_ee^2}$  and the Compton wavelength of the electron $\lambdabar_e=\frac{\hbar}{m_ec}$:

$r_e=\frac{\alpha\lambda_e}{2\pi}=\alpha^2a_0$.

$e^2=1.44\,MeV\,fm$

TODO: pressure ionizzation and electron degeneracy (why eos for H,He)

\subsection{Massa minima innesco fusione H}

Temperatura minima per innesco idrogeno: $T_{min}\approx\SI{1.5e6}{\kelvin}$
\begin{align*}

&P_c=K_{NR}n_e\expy{5/3}+n_ikT_c=K_{NR}[\frac{\rho_c}{m_H}]\expy{5/3}+\frac{\rho_c}{m_H}kT_c\\
&kT_c=A\rho_c\expy{1/3}-B\rho_c\expy{2/3}\\
&kT_{c,max}\propto G^2\frac{m_H\expy{8/3}}{K_{NR}}M\expy{4/3}
\end{align*}

\subsection{Massa massima}

\begin{align*}
&P=P_i+P_e+P_R=\frac{\rho_c}{\mu m_H}kT_c+\frac{1}{3}aT_c^4\\
&P_g=\beta P_c\quad P_R=(1-\beta)P_c
\end{align*}

\subsection{Struttura stellare approssimata}

\begin{itemize}
\item \keyword{Approssimazione zero per $M_r(0)$ e $M_r(R)$}:
Al centro della stella $\TDy{r}{P}=-\frac{GM_r}{r^2}\rho\approx-\frac{4\pi}{3}r^3\rho_c\frac{G}{r^2}\rho_c\approx-\frac{4\pi}{3}rG\rho_c^2$, vicino alla superficie $\TDy{r}{P}\approx-G\frac{M\rho}{r^2}$: se la composizione chimica \'e costante
\begin{align*}
&\TDy{r}{P}\approx-\frac{4\pi}{3}G\rho_c^2r\exp{-r^2/a^2}\\
&P(r)\approx-\frac{2\pi}{3}G\rho_c^2a^2[\exp{-r^2/a^2}-\exp{-R^2/a^2}]
\end{align*}

$4\pi\rho r^2dr=dm$ ci permette di riscrivere equilibrio idrostatico $GM_rdM_r=-4\pir^4dP$:
\begin{align*}
&\frac{1}{2}GM_r^2=-4\pi\int_0^rr'^4\TDy{r}{P}\,dr\\
&M_r=\frac{4\pi a^3}{3}\rho_c\Phi(x)\quad x=r/a\\
&\Phi(x)=6\int_0^xx'^5\exp{-x'^2}\,dx'\approxx^6-3/4x^8+3/10x\expy{10}+...
\end{align*}
per alta $\rho_x$ $a\ll R$ ...:
\begin{align*}
&M\approx M_r=\frac{4\pi\rho_ca^3}{3}\Phi(\frac{R}{a})\approx\frac{4\pi\rho_ca^3}{3}\sqrt{6}\\
&P_c\approx\frac{2\pi}{3}G\rho_c^2a^2\approx[\frac{\pi}{36}]GM\expy{2/3}\rho_c\expy{4/3}
\end{align*}

\item Politropa $n=3/2$: $P_c\propto M\expy{2/3}\rho_c\expy{4/3}$

\item Politropa $n=3$: $P_c\propto M\expy{2/3}\rho_c\expy{4/3}$

\end{itemize}



\section{From proto-star to Pre-MS}

\begin{itemize}
\item Isothermal collapse of MC.
\item Inner region of collapsing cloud becomes denser/optically thicker (first core formation $\tau\approx\SI{1.5e7}{\year}$): role of ambipolar diffusion. Typical mass $M_*=\num{e-2}\msun{}$, $R_*=\SI{5}{\astronomicalunit}$, $\rho\approx\SI{e-10}{\gram\per\cubic\cm}$
\item The first core of molecular hydrogen collapse as soon as T is raised enough to begins molecular dissociation (for $T>2000K$ we have collisional diss)
\begin{align*}
&0=-\frac{1}{2}\frac{GM_*^2}{R_*}+\Delta E_{int}+L_{rad}t\\
&L_{rad}\approx L_{acc}=\frac{\dot{M}GM_*}{R_*}\\
&\Delta E_{int}=\frac{XM_*}{m_H}[\frac{\Delta E_{dis}(H)}{2}+\Delta E_{ion}(H)]+\frac{YM_*\Delta E_{ion}(He)}{4m_H}
\item D ignition: $^2H+^1H\to^3He+\gamma(\SI{5.5}{\mega\ev})$, 
\item radiative core??
\end{align*}
\end{itemize}
