come agire??

{\let\clearpage\relax
\chapter{il palco: scena interna e scena esterna}
}
Problemi della concretezza/La concretezza mi crea problemi: coscienza/soluzioni.

\PartialToc

\section{distrazione}

\begin{itemize}
\item ri-essere come avrei dovuto/sneso colpa
\item \keyword{dimentico obiettivo concretezza}
\item pensieri \keyword{fantasie abitudinarie} interazioni
\item pensiero abitudinario chiuso su se stesso
\item ascolto interpretazione intezioni altro - attesa
\item parole rabbia
\item interlocutore specchio
\item ricordo parole
\item narrazione  innaturale
\item cerco e mi addormento
\end{itemize}

\subsection{impulsi angoscia}

\begin{itemize}
\item non vedo cosa fare
\item riflesso interpretazione altro mie parole
\item non riesco a uccidere l'altro in me in maniera duratura: come narrazione di possibile, desiderii, vita
\item ho uno specchio da analizzare
\item paura-inadeguatezza interazioni
\item Anfoscia fraintendimento - angoscia-rabbia: impotenza per resto sbagliato.
\item \keyword{Sottofondo di pensieri irreali} illusori: \keyword{autonarrazione ironica-sentimentale-rabbiosa}.
\item Teatro chiamare la poggiani: non iesco a trovare lo stato d'animo take no prisonier - mi concentro sulle parole da dire- 
\item fare discorso articolato - fare discorso intimo: \sout{rabbia} \sout{agitazione} \sout{distrazione-dovrei imitare} catena pensieri
\item bugia: \sout{angoscia-semplificazione-autoinganno} \keyword{indipendenza}
\item stato d'animo \keyword{rabbia senza via d'uscita}
\item interlocutore attesa: angoscia reazione, impulso abitudine bias. ... e poi prendo l'insalata. Che tristezza ...
\item ricordo rinfrescato eleonora buongiorno freddo: memoria condizione \keyword{maggio-ottobre13}
\item \keyword{bravo} rossi
\end{itemize}

\subsection{situazioni}

\begin{itemize}
\item \keyword{distrazione/irrequietezza}: ascoltare wwdc e studiare: quante volte rileggo una frase/umore; pomeriggio post incontro paolicchi:
\item \keyword{tempo me (studio)}: angoscia interazioni (aprire la finestra: impulso loquor, non riesco a ''giocare'' non riesco a fottermene), non riesco a parlare senza angoscia in pizzera per chiede se posso pagare domani a causa della gente
\item me narrazione esterna: \keyword{abitudine narrazione come privazione me}
\item partire la mattina presto senza aver dormito: \keyword{lentezza} \keyword{mi parlo}, 
\item stato d'animo telefono alla poggiani con sicurezza vs stato d'animo casa
\item interazioni ostili
\item pomrolla
\item interazione in malafede: soggezione interazione - \keyword{non riesco a oltrepassarti}
\item charging $<-->$ sensazione inadegustezza $->$ consolazione: \keyword{insicurezza} \keyword{inesperienza}
\item stato d'animo - molteplicit\'a -situazioni
\item situazione-nonsochefare sonno seguire senza. ricerca frustrante
\item \keyword{caf\'e}: mi ci va o non mi ci va?
\item situazione-nonsochefare sonno seguire senza. ricerca frustrante
\item situazione angoscia rabbia-parole.
Rabbia commenti soggetti in torno.
\item abitudine interlocutore abitudine vs personale soggetto/oggetto
\item inferiorit\'a-rifiutorifugioabitudine-fuga
\item premura risposta interlocutore
\item interlocutore: rubbish/abitudini-preoccupazione-angoscia
\item inadeguatezza confronto ladro
\item soprassiedo ricerca bici
\item fatica stanchezza trazioni ma le faccio: devo pensare ad un qualcosa
\item Sensazioni paura/rifuggire solitudine: ascoltare radio casa vs bicicletta, doccia silenzio vs riuscire a pensare. SILENZIO interno e esterno: descrizione sensazione e riuscire a trovare pensiero.
\end{itemize}

\section{loquor}

coscienza linguaggio strumento

\subsection{Impotenza di cosa??}

\begin{itemize}
\item non riesco a articolare/bisogno \keyword{riconoscersi} ma condizione confusione (tipo: sono in garage devo rimontare la bicicletta). \keyword{Cosa succede?}
\item sonnolenza che impedisce concretezza / coazione interazioni
\item parlare per parlare
\item cosa dire??
\item condizione loquor
\end{itemize}

\subsection{mamma}

senso di colpa/paura

\subsection{Babbo}

noia/riflesso/coazione

\subsection{bisogno}

\subsection{femmina}

\begin{itemize}
\item soggezione-vuoto-fuga: come agire?
coscenza-obiettivo-preparazione
\end{itemize}

\subsection{''mercato''}

(come prendere coscenza/agire nel ''gretto egoismo''?)

\begin{itemize}
\item prendere coscenza del mio ''gretto egoismo''
\item oggi non lavori: risposta che pensa alla domanda non a sua ontenzione
\end{itemize}


\section{...mi devo preparare..}\label{concretezza}

\begin{itemize}
\item coscienza relazione babbo, mamma
\item relazione/obiettivo esclusivo interazione (garanzia, domande prof/problemi sci)
\item \keyword{leggere informazioni}
\item \keyword{condizione dipendenza aiuto}
\item \keyword{informazioni vs bisogno}
\item per studiare come workout
\end{itemize}

\subsection{ho un obiettivo}

\begin{itemize}
\item rivedere relazione LabNMR.
stimolo coscienza me e controllo angoscia
\item sveglia/realismo
\item seminario sistemi planetari (incontro paolicchi)
workout condizione: cantare con l'emozione solo nella voce
come scegliaere argomenti??
perch\'e ho bisogno di prepararmi prima di chiamare (preparazione per concretezza/indipendenza che \'e utile sempre?)
cosa mi serve? la domanda ha doppio senso: mi serve qualcosa da paolicchi per preparare seminario? cosa voglio imparare/ottenere con ci\'o??
Inserire l'azione in un progetto. Loquor senza impulsi.
Previsioni: Stato d'animo?-Obiettivi? Capitalit\'a obiettivo me, capitalit\'a previsione: io amo il teatro. Coscienza del contesto: preparazione ad ampio spettro sistemi planetarii ovvero scrostare la paura. 
Adesso non riesco a fare a meno di pensare angoscia laboratorio. L'angoscia non mi fa vedere la libert\'a: ''non so fare il compito'' (abitudine: nascondersi, studiare, *controllare il panico), ''non so cosa fare'' (abitudine:  seguire abitudine essere me, impotenza/dovere). Controllo sbalzo umore (esplosione ansia),
Argomenti
\begin{itemize}
\item Molecular clouds: chemistry (atoms, molecules, grain, radiation)
\item Chaos: model for chaotic behaviour of small bodies (JC,NEA) that undergo close encounter with planets, large scale chaos and marginal stability in the solar system (minor bodies, planets, stability of solar system).
\item Three body problem: exoplanets identification (TTV), spacecraft trajectories, restricted problem (EM-spacecraft, SEM, SJA, star-giant-terrestrial, star-exoplanet-exomoon, binary stars-exo-terrestrial/giant, )
\end{itemize}
\item Incontri/incontri babbo/Incontri mamma.
\item Medico. Scopo: divetirmi a raccontare, essere inteligente.
non convincere come al solito ma ''estorcere''
condizione: cosa mi imped\'i di firmare un contratto con mondolibri?? \keyword{bias famiglia - autonomia}
\item anas: barriere fono-assorbenti
chiedo-sollecito
barriere fonoassorbenti
sull'altro lato \'e gi\'a stata montata
E78 - strada del ruffolo 32
stato d'animo: solitamente confusione/ansia
costruzione dopo 10 approvazione progetto
\item incontrare donne
\item Pianigiani rottami
Modus parafango anteriore/paraurti: grigio
Alzacristalli
Vetro anteriore
Smontaggio/prezzi
Obiettivo: a me serve
\item colloquio di \keyword{lavoro}.
Menefreghismo del giudizio: consapevolezza \sout{impulsi}
\item Marchitiello-specializzanda psicologia.
Prima volta che mi sono rivolto ad un medico per il problema di sentirsi svenire. Descrizionesituazione (dopo pranzo caf\'e approx 4 tazzine), sensazione noce sotto lo sterno: svenimento tachicardia, posizione migliore seduto infatti in piedi mi sentivo svenire sdaiato avevo sensazione difficilmente descrivibile (tachicardia-cuore in gola, sensazione fisica oppressione torace) saltuariamente rumore ritmico dal lato sinistro del petto; fase acuta: 3 gg, tendenza tachicardia-affanno: qualche mese.
Malessere: workout prima di cena then svenimento/mancanza d'aria.
Esempio situazione malessere: bike 3 ore, (cioccolata birra) , sforzo gomme, mal di testa, svenimento, cena tachcardia mancanza d'aria, mi sdraio mi passa tutto.
Condizione mi addormento (in coma)/ mi sveglio in coma.
\item chiamo casa
\item sicurezza incontro passera: intenzione-previsione vs attesa aspettazione angoscia dipendenza
\item Criticit\'a: essere ignorato, non capire, non aver loquor ''adulto''
\end{itemize}

\subsection{loquor quale stimolo??}

\begin{itemize}
\item ironia avulsa
\item ascolto-interazione: posizione riflessa
\item devo: obiettivo vs pauraincognitaabitudinefatica
\item post-orgasmo: \'e stato bello, sei bellissima - come controllarsi?
\end{itemize}

\subsection{Workout impulsi: dolore, contatto femminile, rispetto, soddisfazione incoscente}

\begin{itemize}
\item adesso: non ho bisogno di niente
\item conoscere le mie reazioni
\item come accedere a condizione urgenza-azione (concentrazione-determinazione)?
\end{itemize}

\section{interlocutore}

\begin{itemize}
\item incontri: bias essere dipendenza - coazione paura/premio - quasi sempre non sono me
\item Panico/paralisi: argomenti tab\'u con babbo, parlare di scopare, situazione interrogazione insicurezza azione
\item loquor puro: coscienza situazione, scelta intenzione (evitare bias auto-narrazione ironica)
\item Preoccupazione performance (commento altro) vs interesse me (estetico/pratico)
\item cercare/dire; modelli assidui
\item differenza tra interlocutore reazione vs interlocutore ''speriamo di non incontrarci'' vs interlocutore domanda/bisogno
\item tono di voce: importanza bisogno, la solita situazione, non-interlocutore
\end{itemize}

\subsection{tensione attrazione/attenzione pensiero altro}
\begin{itemize}
\item prigione abitudine: vuoto
\end{itemize}


\subsection{Necessit\'a naturali}

di cosa parlare: mi prende l'angoscia
\begin{itemize}
\item Fantasia: scelta, succhiare, fare l'amore
\item devo: non devo sprecare soldi, devo avere soddisfazione
\item paralisi rifiuto paura imbarazzo
\end{itemize}

\subsection{genitori}

\begin{itemize}
\item stato d'animo riempito parole inutili: padre parole a ruota libera - madre parole annoiate.
\item argomenti secondari
\item B: vorrei ma non mi riesce
\item M: non ho voglia diu ascoltare
\item nausea irritazione
\item 7 marzo. B: supeare impulsi: memoria vs ragionamento ascolto
\item 8 marzo: noia ascoltare, reazioni impulsive: controllo; tento piccole imrovvisazioni
\end{itemize}

{\let\clearpage\relax
\chapter{Salvezza fuori dalle parole}
}

\section{Tempo}

\begin{itemize}
\item memo di come uso tempo interno
\item confinare orario/durata condizione interna
\item Rigirarmi/spengere la sveglia/aspettare
\item tempo futuro: non riesco a essere concreto
\item Ora non riesco, ascoltare, soggezione figa
\item tempo passato a ''studiare'' !! (\keyword{prendo il caf\'e per studiare})
\item tempo aspettando vs \keyword{Urgenza}
\end{itemize}

\section{condizione interna}

\begin{itemize}
\item Cosa mi distrae dall'interesse principale?
\item perch\'e non riesco a rispondere alle domande dei genitori?: vergogna personale.
\end{itemize}

\section{ipotesi situazioni}

\subsection{situazioni estreme}

\begin{itemize}
\item bloccato/neutralizzato in azione:
\end{itemize}

\subsection{non ottenere risultato}

\begin{itemize}
\item umore memoria abitudine: il pi\'u delle volte non sono me
\end{itemize}

\section{combattimento}

\begin{itemize}
\item capitalit\'a obiettivo: nitidezza coerenza coscienza
\item identificare la gente che recita a soggetto: non perdere tempo in domande oziose o inseguimenti sterili. \keyword{teatro doppiezza possibilit\'a}
\item buonanotte(sto uscendo da exploit): cerco loquor concreto/ impulso mancata risposta come impulso distrazione: devo faticare per essere me
\item capitalit\'a azione ritmo prontezza concretezza: spingo distributore automatico, confusione folla post doccia/post birra-sono teso non riesco a pensare
\end{itemize}


{\let\clearpage\relax
\chapter{Domande}
}
\PartialToc

\section{Perch\'e non so cosa voglio?}

\begin{itemize}
\item Quali domande necessitano una risposta?!
\item quale utilit\'a hanno queste sezioni??
\item come pensare/seguire progetto non da sonno/rabbia?
\item dove \'e combattere?
\item \keyword{Come interagire realmente con me??}
\item perch\'e ho questa sensazione angoscia quando esco di casa?
\item perch\'e mi passa la sensazione angoscia quando faccio workout boxe?
\item perch\'e torna sensazione angoscia quando mi rendo conto che babbo non viene?
\item cos\'e questo groppo in gola?
\item perch\'e sto ancora perdendo tempo?
(vergogna loquor/ insicurezza femmina/ seghe mentali interpretazione atteggiamento altrui)
\item cosa condiziona il mio umore? Quanto il mio umore condiziona me??
\item Sono fatto da impulsi ciechi?
\item Perch\'e la mattina non ho un motivo per alzarmi??
\item Quanto il mio essere \'e abitudine? Quale?
\item Cosa fare con questi impulsi?
\end{itemize}

\section{pensiero loquor: Aspetto?-Dovrei?-Ora?}

\begin{itemize}
\item Come fare il pagliaccio se entrando al cineforum ma la porta \'e chiusa?
\item perch\'e o sono in ritardo o spreco tempo (dormo ancora, cerco qualcosa,...)?
\item come liberarsi della interiorizzata reazione altrui alle mie azioni/emozioni??
\item Cosa mi mi mette indecisione? Coazione abitudine \keyword{futuro come pensiero coazione} - \keyword{tranquillit\'a compagnia}
\item Perch\'e non riesco a parlare?
\item Perch\'e non riesco a concentrarmi quando incontro estraneo/faccio workout?
\end{itemize}

\section{come? Qual \'e il problema?}

\begin{itemize}
\item Capire questo sonno? Perch\'e mattina?
\item Cosa \'e questo sonno? (Sera)
\item Perch\'e non riesco a trovare ritmo senza angoscia?
\item perch\'e spesso non sono pronto a interazione? (telefonare,medico,tacchinare)
\item Come bias babbo determina essere me (\keyword{desiderio visione realt\'a}) ?
\item Perch\'e mi devo ricordare di essere me?
\item perch\'e legame 
\item Cosa non mi ricordo?
\item Come cercare concretezza??
\item Cosa sono quando mi sveglio alle 6 e non so far altro che continuare a dormire?
\item Perch\'e non ho voglia di fare quello che progetto??
\item Come essere me quando mi addormento??
\item Come concentrarsi su obiettivo attuale??
\end{itemize}

{\let\clearpage\relax
\chapter{tempo interno}
}
\PartialToc

\section{Con gli strumenti attuali}

\subsection{Interazione-me interazione}

\subsection{me abitudine (o scocciato interazione)}

\section{TIME: ? ? ?}

\section{Nitidezza}

\subsection{concretezza me obiettivo}

\begin{itemize}
\item ''Quando avr\'o pensato a obiettivo potr\'o allenarmi a pensarlo''
\item Ridurre iato condizione pi\'u paciosa condizione pi\'u violenta
\item pensare studio e pensare me: stesse difficolt\'a - non riesco ad andare \keyword{oltre quello che ho di fronte}
\end{itemize}

{\let\clearpage\relax
\chapter{coscenza me: malessere disagio incompiutezza}
}

\section{Punti difficili (asettico)}

\subsection{Situazioni paradigmatiche}

\begin{itemize}

\item \keyword{senso di colpa/gratitudine}: le condizioni che mi opccupano e impediscono lucidit\'a

\item prontezza: \keyword{necessit\'a-capitalit\'a loquor}, \keyword{priorit\'a->zavorra}

\item interazione incapacit\'a pensiero

\item Arrossire soggezione figa

\item Interlocutore (coazione ascoltare) condizione sgomento

\item Interazioni incapacit\'a pensiero

\item problema decisionale: dovere vs abitudine. \keyword{quale obiettivo?}

\item attesa telefono babbo: insulsaggini conversazione

\item rabbia post-pizza: stanchezza studio/sonno; descrizione percorso come attesa verso cibo; interzione abitudine non soddisfacente/rabbia/impotenza altri in quanto ostacolo; soddisfazione cibo e noia insofferenza insulsaggine estraneit\'a contesto; attesa pagare confusione.

\item quando estraneo mi da consiglio di buon senso che non seguo per abitudine.

\keyword{dormire in attesa di qualcuno}

\item scontro: imulso piangere / rottura abitudine / mancanza di espeienza essere altri

\item abitudine me noia parlare sotto voce

\item situazione stallo domanda/dipendenza: per questa misura la pianta ha 1/2 anni, \'e molto piccola.

\item \keyword{la situazione padre diventa condizione}.

\item \keyword{la situazione madre diventa condizione}.

\item Le situazioni determinano lo stato d'animo obiettivo

\item loquor dalla noia. \keyword{parlare come dovrei}

\item loquor as dolore errore/fastidio

\item seguire un progetto coscentemente

\item mio stato d'animo pensiero come attesa pesce amo

\item tensione imparare non deve confondersi con abitudine

\item essere abitudine: non si distingue coazione esterna da realt\'a me. Abitudine a seguire contro mio essere sbrigativo/riflesso su me

\item importanza altro: bisogno vs considerazione.

Stato d'animo bici parcheggiata in altro posto

\item non posso aspettare di stare bene

\item memoria dolore

\item dimentico che devo uccidere: il motivo \'e ciao e seguenti

\item \keyword{loquor e difficolt\'a}
\item priorit\'a obiettivo-schema-pensiero
\item \keyword{loquor e difficolt'a}

\item angoscia della condizione

\item pensare discorso

\item quando mi concentro resto solo con \keyword{abitudine me}

\item condizione dopo workout.

\item condizione post birra.

(dipendenza. paura insicurezza.)

volevo fuggire dalla realt\'a ma ho bisogno di qualcos'altro

\item rabbia o fantasie

\item impreparato/ci sono rimasto male

\item coincidere coi miei bisogni - cercare di essere

\item concretezza realt\'a: \keyword{addormentarsi alla guida}, \keyword{dimenticare di aver riattaccato la corrente}

\item mancanza di urgenza della vita

\item perdo tempo decidendo il significato delle azioni

\item alternanza rabbia stanchezza

\item preparazione momento telefonata domande d’insicurezza

\item Incontri inattesi

\item ascolto discorsi conflitto me

\item pensiero intenzione altrui

\item programmazione obiettivi

\item forza nelle uggiosit\'a necessarie

\item distrazione/sonno/noia nella prospettiva abitudine-casa

\item riprendersi da voglia perder tempo

\item pensiero a quello che pensi

\item sonno forte come il mare

\item ricerca noiosa vs memoria riconoscere

\item osservare gli altri che assecondano gli impulsi tab\'u

\item interazione fattorini frigo/angoscia post-incontro

\end{itemize}

\subsection{Condizioni paradigmatiche}

\begin{itemize}

\item birra sonno
\item mi sono trovata a cerchiare a M

\item pensare realt\'a: come si comuncia?

\item esterno: angoscia dovrei - ''mi guardano''

\item \keyword{Come essere vivo?} Perdo tempo, non faccio le cose importanti, aspetto cibo sonno ingurgitare, aspetto che passi angoscia.

\item abitudine bisogno essere creduto/bisogno appoggio
\item sonno mattina: impulso parola senza esperienza altri

\item \keyword{aspetto te}

\item \keyword{sono lento nei collegamenti}

\item condizione/solitudine/birra/non vedo il futuro/perdere tempo

\item \keyword{non riesco a fare con la mente}

\item escluso da ‘’piacere nel dare piacere’’

\item \keyword{distrazione continua}

\item perch\'e faccio: coazione vs voler essere me

Non so cosa voglio vs me

\item dover vendicarsi/pensare parole vs coscenza me

\item non mi ricordo/ho coscenza di quello che voglio: mattina sera momento in cui ci penso 

\item \keyword{concretezza estate madre}

\item relazioni familiarit\'a: \keyword{non ho obiettivi autonomi}

\item \keyword{intenzione asettica}: se mi rivolgi la parola mi chiedo come sia arrivato ad odiarti

\item \keyword{ricordi che emergono} - non farsi sommergere - vedere essere e nulla

\item essere me nel modo che mi esclude

\item auto isolarsi vs intinzione altrui da eradicare

\item impulso uccidere senza concretezza

\item Madre frustrante

\item padre improduttivo

\item Vacanza Puglia (Repubblica). Sveglia. Sonno. Colazione. Camera. Pranzo. Camera. Sonno. Cena. Camera.

\item Coinquilini: mattina aspetto che se ne vadano ...

\end{itemize}

\section{Punti Difficili (correlativo)}

\subsection{maggior tempo non sono me}

\begin{itemize}

\item ascoltare; angoscia indescrivibile - sensazione dovere impotenza tachicardia angoscia paura dipendenza. \keyword{dovere narrazione insicurezza costrizione paura inseguire specchio}

\item rispondere agnese

\item Tutto quello che faccio ha una causa

\item delirio ilare: irreale

\item confronto me - fantasie estranei

\item \keyword{non aver niente da fare di ''pubblico''} - mobbying

\item \keyword{progetto - memoria}

\item pensiero me interagisco: costruire un. personaggio??

\item maggior tempo sogno/deliro su quello che sento!

\item Mi trovai a Marsiglia temporale pochi soldi senza saper che fare ...: abitudine=auto-imprigionarsi??

\item Quando ho bisogno di descrivere la condizione sono ne sono  \keyword{soverchiato}.

\item sonno/noia: auto - distrazione non curanza abitudine inpercepibile

\item carabinieri velocit\'a - panico a fatica riesco a parlare
domandare alla prof. degl'innocenti cose che mi interessano
vs abitudine sonno

\item Stato d’animo per una \keyword{finzione abitudinaria}.

Sensazione interlocutore immaginario-stato d’animo in colpa.

\item accerchiato-non sprecare tempo: libert\'a \sout{rimpianto}, azione concreta

\item dov’\'e la ferocia.

Voglio mangiare ancora.

\item workout me: importanza sensazione - sentire me. Interesse per \keyword{il me in-se}.

Cena-birra-babbo-balbetto

\item le tue parti che ti mettono in \keyword{una parte}

\item \keyword{malafede}-\keyword{sonno-ignoto}-\keyword{angoscia-aiuto}

\item ascoltare -> \keyword{intenzione-obiettivo}

\item (\keyword{abitudini} (\keyword{impulsi} \keyword{pensiero}))

\item quello che \keyword{ignoro/dimentico}

\item \sout{controllare stato d’animo} \keyword{essere altro}

\item nelle situazioni in cui \keyword{non riesco} a essere

\end{itemize}

\subsection{abitudine: da dove vengo}

\begin{itemize}

\item pericoloso sentimento bonario assecondare intenzioni altro

\item panico non sapersi esprimere: discorso involuto

\item mi molce il cuore quando penso che ho sbagliato.

Equazione del moto: cosa causa diffusione termica dell’elio.

\item ‘’ mi fa segno d’accostare’m vs scusi non avevo visto il lampeggiante pensavo ...

\item \keyword{loquor-paura} \keyword{loquor-rabbia}

\item ripetere senza pensare - pensare a caso - pensare a quello che ho di fronte: \keyword{intenzioni altrui come specchio della mia inadeguatezza}.

\item importanza non riconosciuta - do la stessa importanza a sensazioni, modi d’essere, progetti - non capisco quello che ho scritto 

\item \keyword{loquor-dipendenza}, \keyword{loquor-frustrazioneimpotenza}, \keyword{loquor-reflecireflexi}

\item \keyword{Paura (giustificata)}: \sout{panico - inesperienza}, \sout{rabbia - impotenza} \keyword{impulso-esperienza}.

\item Auto-contemplazione: \keyword{contemplazione memoria autore me}

\item ascolto-pensiero parole-angoscia interazione bisogno-\keyword{abitudine fuga}

\item \sout{fantasia scenica} \keyword{azione e reazione}

\item \keyword{freddo}: voglia abitudine stanchezza. Non faccio ginnastica: pesantezza.

\item combattimento: pensiero-parola -> insicurezza-rabbia; pensiero punchball tensione muscolare.

\item Attesa Aspettazione-Soddisfazione: \keyword{perch\'e mi sveglio?}

Paura-Insicurezza-Impotenza Rabbia-Impulso-Orgoglio: \keyword{inesperienza}.

\item stato d’animo \keyword{possibilit\'a}: \sout{angoscia atavica}, \sout{auto soddisfazione}, memoria-obiettivo.

\item \sout{svenimento/rabbia/impulsi ciechi} combattimento

\item Cos’e l’abitudine: quello che sono \keyword{quasi sempre} - vitale evitare \keyword{impreparazione}

\item interazione: \keyword{cose intime} (dolore - angoscia), (pensiero loquor??), abitudine.

\item proiettato \keyword{interazione}:

\keyword{domanda interazione}: 

\item sveglia mattina voglio \keyword{aspettare} ...  \keyword{\textgreek{dietriya}}

\end{itemize}
