\chapter{stahler palla (43) - ISM}

\begin{itemize}
\item Galactic gas. HI: $E=\SI{13.6}{\ev}n\expy{-2}$: UV Lyman series $2\to1$-$\Lya$ (\SI{1216}{\angstrom}), $3\to1$-$\Lyb$ (\SI{1026}{\angstrom}), balmer serie-$3\to2$ ($\Ha$: \SI{6563}{\angstrom}). Hyperfine traansition from higher excited parallel e-spin and p-magnetic moment (some fraction of collisionally excited H deexcite through photon emission) - convert column density to volume density require knowledge of location of emitting gas: overall shift in peak intensity steam for diff rotation of galaxy which is known thus analysis of emission profile in many directions yields number density $n_{HI}$; 21cm-absorption profile probes gas temperature. HI confined in clouds +warm unconfined component.
Molecular Gas: CO is strong emitter - molecular clouds optically thick to \SI{2.6}{\milli\meter} line (trace total $H_2$ in the cloud); $H_2$ distribution in galxy peaks at \SI{6}{\kilo\parsec} related to star formation and little outside \SI{10}{\kilo\parsec}, more confined to g-plane.
HII regions around O/B stars: stellar Lyman continuum ionized for several parsec - emission from heavier allow composition reconstruction / absorption from star behind HI cloud: HII of M42 Z depleted (segregated in dust).
Phases of atomic components: cooling (low density $\Lya$ line, higher density CII \SI{158}{\micro\meter} transition) heating (starlight causes \Pelectron emission from dust grains) terms determine T-eq (for \SI{1}{\per\cubic\cm} \SI{e4}{\kelvin}) what state matches known interstellare pressure?
\item Interstellar dust. Extinction $m_{\lambda}=M_{\lambda}+5\log{(\frac{r}{\SI{10}{\parsec}})}+A_{\lambda}$, color excess $E_{12}=A_{\lambda_1}-A_{\lambda_2}$ both proportional to column density of dust on LOS ($\frac{E_{\lambda-V}}{E_{B-V}}=\frac{A_{\lambda}}{E_{B-V}}-\frac{A_V}{E_{B-V}}=\frac{A_{\lambda}}{E_{B-V}}-R$), assuming $A\approx0$ for long wavelength.
$I_{\nu}\Delta\nu\Delta A\Delta\Omega$ is energy per unit time between $[\nu,\nu+\Delta\nu]$ propagating normal to area $\Delta A$ within $\Delta\Omega$n (flux across surface with normal $\hat{z}$ is energy per unit area per unit time $F_{\nu}=\int\mu I_{\nu}\,d\Omega$, $\mu=\hat{n}\cdot\hat{z}$ cosine between propagation direction and normal), total specific energy density $u_{\nu}$ and mean intensity $J_{\nu}$ - $\Delta I_{\nu}=-\rho\kappa_{\nu}I_{\nu}\Delta s+j_{\nu}\Delta s$; at $\lambda$ of negligible dust emssion $I_{\lambda}(r)=I_{\lambda}(R_*)\exp{-\Delta\tau_{\lambda}}$ and flux at r much greater than $R_*$ ($\mu\approx1$) and $\Delta\Omega=\pi\frac{R_*^2}{r^2}$: $F_{\lambda}=\pi I_{\lambda}(R_*)(\frac{R_*}{r})^2\exp{-\Delta\tau_{\lambda}}$ - $A_{\lambda}=2.5\log{e}\Delta\tau_{\lambda}$.
Thermal dust emission $T_d$ different from gas ($\frac{\nu_M}{T}=\SI{5.88e10}{\hertz\per\kelvin}$, $\lambda_MT=\SI{0.29}{\cm\kelvin}$). Efficiency of extinction: $\rho\kappa_{\nu}=n_d\sigma_dQ_{\nu}$, $Q_{\lambda}=0.14\frac{A_{\lambda}}{E_{B-V}}$ - $Q_{\lambda}\propto\invers{\lambda}$ in optical, local maximum at longer $\lambda=\SI{10}{\micro\meter}$, peak in UV \SI{2200}{\angstrom} - at far IR and mm ISM is transparent so we observe emission from heated dust clouds. Received flux $F_{\lambda}=B_{\lambda}(T_d)\Delta\Omega\Delta\tau_{\lambda}$ that yields $\Delta_{\lambda}(\ll1)$ ($T_d$and $A_V$ determination is problematic so $Q_{\lambda}$ is poorly determined): $\SI{30}{\micro\meter}\leq\lambda\SI{1}{\milli\meter}$ $\lambda\expy{-\beta}$ - physical agglomeration in denser regions. Gemoteric cross-section per H atom $\Sigma_d=\frac{n_d\sigma_d}{n_H}=\frac{(Ln_d)\sigma_d}{(Ln_H)}$.
\item Molecular clouds. Diffuse clouds: H atomic/molecular, $A_V\approx1$ - are minor fraction of IS gas and don't produce stars; observation of \SI{2.6}{\milli\meter} line of $^{12}C^{16}O$ show that OB association are related with giant molecular clouds ($A_V\approx2$, $n_H\approx\SI{100}{\per\cubic\cm}$, $L\approx\SI{50}{\parsec}$, $T\approx\SI{15}{\kelvin}$, $M\approx\num{e5}\msun$) - map of molecular cloud from doppler effect given known law of galactic rotation. If line is optically thick we can use rarer isotopes. Volume of complexes are filled by dense clumb and less dense atomic/molecular constituent. Formed through clamps accumulation prob. as gas inflow in arms potential well (clustering of clouds along spiral arms) and condense thanks to self-shielding.
\item Magnitude of ''Random'' velocity dispersion of clumps within complex consistency with gravitational field. EOM $\rho\frac{D\vec{u}}{Dt}=-\nabla P-\rho\nabla\phi+\frac{1}{c}\vecp{j}{B}\to-\nabla P-\rho\nabla\phi+\frac{1}{4\pi}(\vec{B}\cdot\nabla)\vec{B}-\frac{1}{8\pi}\nabla|B|^2$ third term repr. tension associted with curved field lines and last gradient of scalar magnetic pressure: scalar produc of above eq integrated over V give virial theorem $\frac{1}{2}\TtwoDy{t}{I}=2T+2U+W+M$
\end{itemize}

\chapter{stahler palla (267)- From clouds to stars}

\begin{itemize}
\item Molecular clouds (pg 71). 
\item Collapse is localized within lorge complexes - since most molecular gas is not collapsing we investigate equilibrium of forces under which clouds can last over long period
\item Equilibrium of thermal pressure and self-gravity: isothermal first approximation for dense core. Lane-Eden equation: gravitatonal stability and critical lenght scale using isothermal Lane-Emden model (for stable cloud increases of $P_0$ causes increase of internal pressure: clouds of low density constrast are confined by external pressure, all clouds with $\frac{\rho}{\rho_c}14.$ at right of first maximum are gravitationally unstable - $M_{BE}=\frac{m_1a_T^4}{P_0\expy{1/2}G\expy{3/2}}$). Jeans classics analysis of self-graviting waves propagating through uniform medium: $\lambda_J=\sqrt{\frac{\pi c_s^2}{G\rho_0}}$ - dense core/Bok globe are close to Instability
\item rotation support. Dense cores have measured rotation rate (\SI{1.27}{\cm} line of $NH_3$ and others - ch3), and gradient in radial velocity from CO mapping of GMC. Cyl symm-sys specific AM about z $j=ou_{\phi}=r\sin\thetau_{\phi}$: centrifugal force per unit mass $j^2/o^3$ which have to be balanced in EOM, centrifugal potential $\Phi_{cen}=-\int_0^oj^2/o^3\,do$ so we can extendend the non rotating case. Numerical methods: trial density. 
\end{itemize}