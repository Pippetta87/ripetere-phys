\begin{frame}{Accrescimento dei planetesimi: collisioni.}
\begin{itemize}
\item $r_{pl}<b<\frac{Gm_{pl}}{v_r^2}=b_0$: incontro ravvicinato ($|U_{max}|>E_{kin}^{\infty}$ l'energia potenziale nel momento di massima interazione sovrasta energia cinetica a infinito).
\item $b>(\frac{m_{pl}}{3\msun{}})\expy{\frac{1}{3}}a=d_{Hill}$: passaggio a grande distanza.
\item $b_0<b<d_{Hill}$: deflessioni di direzione e verso casuali.
\end{itemize}
\begin{block}{Scattering $b>r_{pl}$: variazione velocit\'a relativa}
\begin{equation*}
\delta v_r=\underbrace{\frac{Gm_{pl}}{b^2}}_{\exv{a}}\underbrace{\frac{2b}{v_r}}_{\exv{\tau_{int}}}=\frac{2Gm_{pl}}{bv_r}
\end{equation*}
\end{block}
\begin{equation*}
b_0<b<d_{Hill}:\quad [\TDy{t}{v_r^2}]_{enc}=\int_{r_{pl}}^d2\pi bv_r\frac{\sigma n}{m_{pl}v_r}(\frac{2Gm_{pl}}{bv_r})^2\,db
\end{equation*}
\begin{block}{Urti anelastici: impatti}
\begin{equation*}
[\TDy{t}{v_r^2}]_{imp}=\pi r_{pl}^2(\frac{\sigma n}{m_{pl}v_r})v_r(-v_r)
\end{equation*}
\end{block}
\end{frame}