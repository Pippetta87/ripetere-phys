\documentclass[main.tex]{subfiles}
 
\begin{document}

\section{Test ipotesi e GOF}

\keyword{Prob. Errore tipo I $\alpha$}: scarto $H_0$ quando \'e vera (falso positivo, loss,size,taglia,livello di significativit\'a $\alpha=\prob{(x\in C|H_0)}=\prob_0{(x\in C)}$.

\keyword{Prob. Errore tipo II $\beta$}: probabilit\'a di accettare erroneamente $H_0$ (falso negativo, contaminazione), $\prob{(x\in\overline{C}|\non{H_0})}=\beta$. Potenza del test $1-\beta=\prob{(x\in C|H_1)}$ \'e la capacit\'a di distinguere ipotesi diverse da $H_0$.

Fissato $\cl=\alpha$ impongo la condizione \[P_{\theta}[\delta(X)=d_1]=P_{\theta}[X\in S_1]\leq\alpha,\quad \forall\theta\in\Omega_H\]
minimizzo $P_{\theta}[\delta(X)=d_0]$ $\forall\theta\in\Omega_K$ equiv. to maximize
\[1-\beta=P_{\theta}[\delta(X)=d_1]=P_{\theta}[X\in S_1], \forall\theta\in\Omega_K\]
Di solito la condizione precedente implica size of test $\sup_{\Omega_H}{P_{\theta}[X\in S_1]}=\alpha$.

\section{Ipotesi semplici: UMP - Lemma NP}

\end{document}