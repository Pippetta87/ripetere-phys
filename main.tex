\makeatletter
\let\@starttocorig\@starttoc
\makeatother

\documentclass[oneside,20pt,fleqn,extrafontsizes]{memoir}

%%% color definition names: xcolor
\usepackage[usenames,dvipsnames]{xcolor}
\definecolor{bulgarianrose}{rgb}{0.28, 0.02, 0.03}
\definecolor{ashgrey}{rgb}{0.7, 0.75, 0.71}
\definecolor{red(ryb)}{rgb}{1.0, 0.15, 0.07}
\definecolor{wocrit}{rgb}{0.75, 0.25, 0.0}

%%% Import other packages%
\usepackage{usepkg}

%% import moodded thing
\usepackage{localF}
\usepackage{mathOp}
%%%%%%%%%%%%%%%%%% refname, titletoc, titlesec setting
\usepackage{titleT-ref}%toc depth

%%%%%%%%%%%% Hyperref package
\usepackage{hyperref}
\hypersetup{
    colorlinks,
    citecolor=black,
    filecolor=black,
    linkcolor=black,
    urlcolor=black
}

%%%%%%%%%%%%%%%%%%%%%%%%
%%%%%%%%%%%%%%%%%Geometry package
\usepackage{mygeometry}
%%%%%%%
\usepackage{fancypkg}%%%import fancy package with memoir correction

%%%%%%%%%%%%%%%%%% setlength
\usepackage{mylength}
\linespread{0.5}

%%%%%%% META
\title{Esami Luglio 19}

%%import makeidx
%\usepackage{makeidx}

%%% COUNTERS
\setsecnumdepth{subsection}
\setcounter{tocdepth}{-1}       % chapter
\newcounter{cherrychapter}%[chapte]
\setcounter{cherrychapter}{1}
\newcounter{partworkouts}%[chapte]
\setcounter{partworkouts}{1}

%%%% iimport functions
\usepackage{functions}
\usepackage{sources}
%%%%%%%%%%%%%%%%%%%% MULTINDExssss (number of chapters)
%\makeindex[1]%biasmomentumoggi]%chapters
%\makeindex[2]%meditazione]
%\makeindex[3]%nitidezza]
%\makeindex[4]%teatro]
\setsecnumdepth{subsection}
\settocdepth{chapter}
 
\begin{document}%BEGIN
\pagestyle{mystyle}%% mystyle defined in usepkg
\renewcommand*{\contentsname}{\label{toc}{Table of Contents}}%%\printcontents without head/hypereff
%\makeatletter
%\renewcommand*{\@tocmaketitle}{\label{toc}{Table of Contents}}
%\makeatother
\maketitle
\startcontents[chapters]

\tableofcontents*

\part{Fisica nucleare}
\chapter{BB nucleosynthesis}
\PartialToc

\section{Proton capture on light elements}

\begin{itemize}
\item $D(^1H,\gamma)^3He$,$T_D\approx \SI{1e6}{\kelvin}$
\item $^7Li(^1H,^4He)^4He$, $T_{Li}\approx\SI{3e6}{\kelvin}$ 

\end{itemize}

\chapter{Nuclear burning in stars}
\PartialToc

\section{H burning: pp chain}

\section{H burning: CNO cycle}

\section{He burning}

\section{C-burning}




\part{Fundamental interactions}\label{intfon}
\chapter{opacity sources}

\chapter{Evapopration ??}

\part{Analisi statistica dati}\label{asd}
disuguaglianza di Chebychev X ha media $\mu$ e var $\sigma^2$ e $\lambda>0$:
\[\prob{(|X-\mu|<\lambda\sigma)}\geq1-\frac{1}{\lambda^2}\]

\section{Propriet\'a stimatori}

\begin{itemize}
\item 
\end{itemize}

\section{Stimatore mle di varianza segnale gaussiano con rumore gaussiano}

statistica suff: $s^2=\sum_ix_i^2$
stima mle $\hat{\sigma}^2=\max{(s^2/N-\sigma_0,0)}$

var usando funzione GM della gaussiana
(momento n=2 integrato da $N\sigma_0$
(CRC PRESS 2000, pg 408)


\part{Evoluzione componente solida in dischi protoplanetari}\label{ppd}
\begin{frame}{Accrescimento dei planetesimi: collisioni.}
\begin{itemize}
\item $r_{pl}<b<\frac{Gm_{pl}}{v_r^2}=b_0$: incontro ravvicinato ($|U_{max}|>E_{kin}^{\infty}$ l'energia potenziale nel momento di massima interazione sovrasta energia cinetica a infinito).
\item $b>(\frac{m_{pl}}{3\msun{}})\expy{\frac{1}{3}}a=d_{Hill}$: passaggio a grande distanza.
\item $b_0<b<d_{Hill}$: deflessioni di direzione e verso casuali.
\end{itemize}
\begin{block}{Scattering $b>r_{pl}$: variazione velocit\'a relativa}
\begin{equation*}
\delta v_r=\underbrace{\frac{Gm_{pl}}{b^2}}_{\exv{a}}\underbrace{\frac{2b}{v_r}}_{\exv{\tau_{int}}}=\frac{2Gm_{pl}}{bv_r}
\end{equation*}
\end{block}
\begin{equation*}
b_0<b<d_{Hill}:\quad [\TDy{t}{v_r^2}]_{enc}=\int_{r_{pl}}^d2\pi bv_r\frac{\sigma n}{m_{pl}v_r}(\frac{2Gm_{pl}}{bv_r})^2\,db
\end{equation*}
\begin{block}{Urti anelastici: impatti}
\begin{equation*}
[\TDy{t}{v_r^2}]_{imp}=\pi r_{pl}^2(\frac{\sigma n}{m_{pl}v_r})v_r(-v_r)
\end{equation*}
\end{block}
\end{frame}
\chapter{Gravitational N-body problems}

\section{two-body}

\section{3-body}

\chapter{Accretion disk}

\section{Enhanced viscosity: evolution of surface density}

$\eta$ Viscosity= Stress/(strain/time), (strain/time)=shear rate - shear stress $\tau_{y,x}=-\mu\TDy{y}{u_x}$


\section{Gas-planet interaction}
 
\subsection{Impulsive approx}
planet migration (Lubow ida 10): pg 3
Particle $\Omega(r)$ approach planet $\Omega_p=\Omega(a)$ on a circular orbit: small deflection $r-a\gtrsim R_H$
 
 
 
 
 
 
 
 

\part{Galaxy: cluster, stars, molecular clouds}\label{stars}
\section{Ordini di grandezza}

\subsection{Misure adimensionali interazioni}

Potenziale tra 2 nucleoni $-Gm_H^2/r$ con $r=\frac{h}{2\pi m_Hc}$ (Pindet): in unit\'a $m_Hc^2$ $\alpha_G=\frac{2\pi Gm_H^2}{hc}\approx\num{e-39}$.

$\alpha_{SF}=\frac{e^2}{(4\pi\epsilon_0)\hbar c}$.

The ratio of three characteristic lengths:

the classical electron radius $r_e=(\frac{1}{4\pi\epsilon_0})\frac{e^2}{m_ec^2}$, the Bohr radius $a_0=(4\pi\epsilon_0)\frac{\hbar^2}{m_ee^2}$  and the Compton wavelength of the electron $\lambdabar_e=\frac{\hbar}{m_ec}$:

$r_e=\frac{\alpha\lambda_e}{2\pi}=\alpha^2a_0$.

$e^2=1.44\,MeV\,fm$

TODO: pressure ionizzation and electron degeneracy (why eos for H,He)

\subsection{Massa minima innesco fusione H}

Temperatura minima per innesco idrogeno: $T_{min}\approx\SI{1.5e6}{\kelvin}$
\begin{align*}

&P_c=K_{NR}n_e\expy{5/3}+n_ikT_c=K_{NR}[\frac{\rho_c}{m_H}]\expy{5/3}+\frac{\rho_c}{m_H}kT_c\\
&kT_c=A\rho_c\expy{1/3}-B\rho_c\expy{2/3}\\
&kT_{c,max}\propto G^2\frac{m_H\expy{8/3}}{K_{NR}}M\expy{4/3}
\end{align*}

\subsection{Massa massima}

\begin{align*}
&P=P_i+P_e+P_R=\frac{\rho_c}{\mu m_H}kT_c+\frac{1}{3}aT_c^4\\
&P_g=\beta P_c\quad P_R=(1-\beta)P_c
\end{align*}

\subsection{Struttura stellare approssimata}

\begin{itemize}
\item \keyword{Approssimazione zero per $M_r(0)$ e $M_r(R)$}:
Al centro della stella $\TDy{r}{P}=-\frac{GM_r}{r^2}\rho\approx-\frac{4\pi}{3}r^3\rho_c\frac{G}{r^2}\rho_c\approx-\frac{4\pi}{3}rG\rho_c^2$, vicino alla superficie $\TDy{r}{P}\approx-G\frac{M\rho}{r^2}$: se la composizione chimica \'e costante
\begin{align*}
&\TDy{r}{P}\approx-\frac{4\pi}{3}G\rho_c^2r\exp{-r^2/a^2}\\
&P(r)\approx-\frac{2\pi}{3}G\rho_c^2a^2[\exp{-r^2/a^2}-\exp{-R^2/a^2}]
\end{align*}

$4\pi\rho r^2dr=dm$ ci permette di riscrivere equilibrio idrostatico $GM_rdM_r=-4\pir^4dP$:
\begin{align*}
&\frac{1}{2}GM_r^2=-4\pi\int_0^rr'^4\TDy{r}{P}\,dr\\
&M_r=\frac{4\pi a^3}{3}\rho_c\Phi(x)\quad x=r/a\\
&\Phi(x)=6\int_0^xx'^5\exp{-x'^2}\,dx'\approxx^6-3/4x^8+3/10x\expy{10}+...
\end{align*}
per alta $\rho_x$ $a\ll R$ ...:
\begin{align*}
&M\approx M_r=\frac{4\pi\rho_ca^3}{3}\Phi(\frac{R}{a})\approx\frac{4\pi\rho_ca^3}{3}\sqrt{6}\\
&P_c\approx\frac{2\pi}{3}G\rho_c^2a^2\approx[\frac{\pi}{36}]GM\expy{2/3}\rho_c\expy{4/3}
\end{align*}

\item Politropa $n=3/2$: $P_c\propto M\expy{2/3}\rho_c\expy{4/3}$

\item Politropa $n=3$: $P_c\propto M\expy{2/3}\rho_c\expy{4/3}$

\end{itemize}



\section{From proto-star to Pre-MS}

\begin{itemize}
\item Isothermal collapse of MC.
\item Inner region of collapsing cloud becomes denser/optically thicker (first core formation $\tau\approx\SI{1.5e7}{\year}$): role of ambipolar diffusion. Typical mass $M_*=\num{e-2}\msun{}$, $R_*=\SI{5}{\astronomicalunit}$, $\rho\approx\SI{e-10}{\gram\per\cubic\cm}$
\item The first core of molecular hydrogen collapse as soon as T is raised enough to begins molecular dissociation (for $T>2000K$ we have collisional diss)
\begin{align*}
&0=-\frac{1}{2}\frac{GM_*^2}{R_*}+\Delta E_{int}+L_{rad}t\\
&L_{rad}\approx L_{acc}=\frac{\dot{M}GM_*}{R_*}\\
&\Delta E_{int}=\frac{XM_*}{m_H}[\frac{\Delta E_{dis}(H)}{2}+\Delta E_{ion}(H)]+\frac{YM_*\Delta E_{ion}(He)}{4m_H}
\item D ignition: $^2H+^1H\to^3He+\gamma(\SI{5.5}{\mega\ev})$, 
\item radiative core??
\end{align*}
\end{itemize}


\stopcontents[chapters]
\end{document}