\documentclass[main.tex]{subfiles}
 
\begin{document}

\chapter{Comportamento materia densit\'a temperature stellari}
\PartialToc

\section{EOS}

\subsection{Fully ionized perfect gas}

\keyword{perfect gas}: potential energy between particle negligible respect kinetic energy and de Broglie $\lambda_{dB}=\frac{h}{p}\ll\lambda$ mean distance between particles - maxwell distro with mean kinetic energy $\frac{3}{2}kT$:
\begin{align*}
&kT>\frac{Z^2e^2}{d}\\
&\frac{4\pi}{3}nd^3=\frac{4\pi\rho}{3\mu m_H}d^3=1\\
&d\approx(\frac{\mu m_H}{\rho})\expy{1/3}\\
&\frac{h}{p}=\frac{h}{\sqrt{2mkT}}\ll d
\end{align*}
\keyword{Free energy per unit mass of perfect gas}:
\begin{align*}
&F=-kT\sum_sN_s[\frac{3}{2}\ln{T}-\ln{(N_s\rho)}+\ln{G_s}+1+\ln{g_s}]+F_{int}
\end{align*}
$g_s$ statistical weight (number of state with same energy), $G_s=(2\pi km_s/h^2)\expy{3/2}$, $F_{int}$ energia libera particelle con stati legati

\subsection{Non-ideal corrections: Interazione coulombiana}

Corrections parametrized by $\Gamma=\frac{(Ze)^2}{dkT}$: when density suff. low $\Gamma\approx0$; for fully ionized mixture
\begin{align*}
&\Gamma=\frac{\exv{Z}\expy{1/3}e^2}{d'kT}\exv{Z\expy{5/3}}\\
&d'=(\frac{4\pi\rho}{3\mu_im_H})\expy{-1/3}
\end{align*}
La principale correzione che tiene conto dell'interazioni tra particelle \'e dovuta alle interazioni coulombiane.

\begin{workout}[Correzioni interazioni coulombiane- vedi schermaggio]

\end{workout}

La presenza dello ione $Z_i$ altera la distribuzione di cariche quindi devo determinare il potenziale $\phi_i$ generato da $Z_i$ e dalla distribuzione di cariche attorno; usando la formula di Boltzmann la densit\'a delle particelle con carica Z risulta
\begin{equation}\label{eq:Bdistroelectricpot}
n_Z=\overline{n}_Z\exp{-\frac{Ze\phi}{kT}}
\end{equation}
con $\overline{n}_Z$ densit\'a numerica della particella di carica $Z$ in assenza di $Z_i$.

L'equazione di Poisson per $\phi$ \'e
\begin{equation}\label{eq:poissonscreened}
\nabla^2\phi=-4\pi e\sum_Z Zn_Z-4\pi e\sum_i Z_i\delta(\vec{r}-\vec{r}_i)
\end{equation}

Nel regime di schermaggio debole, dove l'energia coulombiana \'e molto minore dell'energia termica $e\phi\ll KT$, espando $n_Z$ (\eqref{eq:Bdistroelectricpot}) al prim'ordine quindi in \eqref{eq:poissonscreened} rimane il termine lineare in $\phi$.

Introduco il raggio di Debye:
\begin{equation}
\frac{1}{r_D^2}=\frac{4\pi e^2}{kT}\sum Z^2\overline{n}_Z=\frac{4\pi e^2}{kT}N_A\zeta,\ \zeta=\sum_{i}(Z_i^2+Z_i)\frac{\rho X_i}{A_i}\label{eq:debyeradius}
\end{equation}
quindi dall'equazione \eqref{eq:poissonscreened} ottengo il potenziale generato dallo ione $Z_i$ in $\vec{r}_i$ a distanza $|\vec{r}-\vec{r}_i|=r_i$ come soluzione di:
\begin{equation}\label{eq:poissonspheric}
\frac{r_D^2}{r_i}\TtwoDy{r_i}{(r_i\phi_i)}=\phi_i
\end{equation}
con
\begin{equation}
\phi=\sum_i\phi_i
\end{equation}

\begin{workout}[Raggio di debye e degenerazione elettronica]
Il rapporto fra densit\'a elettronica e quella imperturbata \'e
\begin{equation}
\midfrac{f(\psi-\midfrac{U_e(r)}{KT})}{f(\psi)}
\end{equation}
dove $U_e(r)$ \'e l'energia di interazione di un elettrone
\end{workout}

La soluzione di \eqref{eq:poissonspheric} \'e
\begin{equation}\label{eq:screenedpotential}
\phi_i=\frac{Z_ie}{r_i}\exp{-\midfrac{r}{r_D}}
\end{equation}

[formula corretta potenziale elettrostatico generato da nuvola elettronica attorno a $Z_i$]

\begin{align*}
&\nabla^2V_i=-4\pi e\sum Zn_Z\approx\frac{1}{r_D^2}V_i\intertext{da cui ottengo il potenziale generato dalla nube di cariche attorno a Z}
&\phi_Z=-\frac{eZ}{r_D}
\end{align*}
con $r_D$ raggio di Debye definito da:
\begin{equation*}
\frac{1}{r_D^2}=\frac{4\pi e^2}{kT}\sum Z^2\overline{n}_Z=\frac{4\pi e^2}{kT}N_A\zeta,\ \zeta=\sum_{i}(Z_i^2+Z_i)\frac{\rho X_i}{A_i}\label{eq:debyeradius}
\end{equation*}

[correzioni $u_c/P_c$: notazione consistente.]
Le correzioni dovute alle interazioni coulombiane sono
\begin{align}
&u=\frac{3}{2}\frac{\gasconstant{}T}{\mu}+u_c,\ \rho u_c=\frac{1}{2}\sum_ZeZ\overline{n}_Z\phi_Z=-e^3\sqrt{\frac{\pi\rho}{kT}}(N_A\zeta)\expy{\frac{3}{2}}\\
&P=\frac{\rho}{\mu}kT+P_c,\  P_c=\frac{1}{3}u_c
\end{align}

Le correzioni dovute alle interazioni coulombiane sono
\begin{align}
&u=\frac{3}{2}\frac{\gasconstant{}T}{\mu}+u_c,\ \rho u_c=\frac{1}{2}\int\phi(\vec{r})\rho_c(\vec{r})\,d^3r\\
&P=\frac{\rho}{\mu}kT+P_c,\  P_c=\frac{1}{3}\rho u_c
\end{align}

\section{Ordini di grandezza}

\subsection{Misure adimensionali interazioni}

Potenziale tra 2 nucleoni $-Gm_H^2/r$ con $r=\frac{h}{2\pi m_Hc}$ (Pindet): in unit\'a $m_Hc^2$ $\alpha_G=\frac{2\pi Gm_H^2}{hc}\approx\num{e-39}$.

$\alpha_{SF}=\frac{e^2}{(4\pi\epsilon_0)\hbar c}$.

The ratio of three characteristic lengths:

the classical electron radius $r_e=(\frac{1}{4\pi\epsilon_0})\frac{e^2}{m_ec^2}$, the Bohr radius $a_0=(4\pi\epsilon_0)\frac{\hbar^2}{m_ee^2}$  and the Compton wavelength of the electron $\lambdabar_e=\frac{\hbar}{m_ec}$:

$r_e=\frac{\alpha\lambda_e}{2\pi}=\alpha^2a_0$.

$e^2=1.44\,MeV\,fm$

TODO: pressure ionizzation and electron degeneracy (why eos for H,He)

\section{neutrino processes}

\end{document}