\documentclass[main.tex]{subfiles}
 
\begin{document}

\chapter{Questo mese: Ottobre}
\PartialToc

\section{Stellare}
\begin{itemize}
\item \href{http://campus.unibo.it/111203/1/pop1_2013.pptx.pdf}{popolazioni: osservazioni, et\'a, distanza}
\item Succo di stellare
\end{itemize}

\section{ASD}

\section{Interazioni fondamentali: Lun 14-16, mar 14-16, ven 11-13}

\section{Computational methods: Lun 9-11 e 16-18, Gio 9-11}

\section{Processi astrofisici: lun 16-18, mer 11-13}

\section{Astrofisica osservativa: mar 16-18, gio 16-19}

\section{Sistemi planetarii}

\section{Decadimento e nuova carriera in laurea magistrale}
\begin{itemize}

\item Con cosa sostituire il fibonacci a Pisa? centri commerciali attrezzati con tavoli e prese di corrente ();  
McDonalds (); fattibile lavarsi con water bag??
Progetto alternativo a unipi? progetto alternativo a universit\'a?

\end{itemize}

\chapter{Mese prossimo: Novembre}


\secion{ASD: GOF-test}

\begin{itemize}
\item Succo di ASD
\item \href{http://pdg.lbl.gov/2013/reviews/rpp2013-rev-statistics.pdf}{statistics-pdg 2013}
\item \href{https://en.m.wikipedia.org/wiki/Monotone_likelihood_ratio}{Monotone likelihood ratio}
\item \href{https://pdfs.semanticscholar.org/b030/c65f8bd32e141ef1a28f7e289921786fadcf.pdf}{MM}
\item \href{https://web.stanford.edu/class/archive/stats/stats200/stats200.1172/Lecture22.pdf}{Generalized LR test}
\item \href{https://www.google.com/search?q=gof%20binned%20data%20likelihood%20ratio&ie=utf-8&oe=utf-8&client=firefox-b-m#sbfbu=1&pi=gof%20binned%20data%20likelihood%20ratio}{gof binned data likelihood ratio}
\item \href{https://agenda.infn.it/event/16360/contributions/33703/attachments/63753/76855/cowan_paestum_2019.pdf}{Cowan-estimators}
\item \href{https://www.physik.hu-berlin.de/de/gk1504/block-courses/autumn-2010/program_and_talks/Verkerke_part3}{Parameter etimation - $\chi^2$ and L}
\item \href{https://arxiv.org/pdf/physics/0509008.pdf}{GOF in likelihood fit}
\item \href{https://physique.cuso.ch/fileadmin/physique/document/jameschap7.pdf}{fit histogram}
\end{itemize}

\chapter{Mese scorso: settembre}

\begin{itemize}
\item 
\end{itemize}

\end{document}