\documentclass[main.tex]{subfiles}
 
\begin{document}

\chapter{Accretion disk}
%\PartialToc

\section{Enhanced viscosity: evolution of surface density}

$\eta$ Viscosity= Stress/(strain/time), (strain/time)=shear rate - shear stress $\tau_{y,x}=-\mu\TDy{y}{u_x}$

\section{Temperature profile}
\begin{itemize}
\item stellar irradiation: $P_i=\pi\epsilon_iR^2F_{\odot}$, $P_o=4\pi\epsilon_oR^2\sigma T_e^4$ - $T_{equi}=280\si{\kelvin}(\frac{a}{\SI{1}{\astronomicalunit}})^{-1/2}(\frac{M_*}{\msun})$
\item viscous heating
\end{itemize}
The torque $\PDy{r}{G}\Delta r$ compie lavoro  ad un rate
\begin{equation*}
\Omega\PDy{r}{G}\Delta r=[\PDof{r}(G\Omega)-G\TDy{r}{\Omega}]\Delta r
\end{equation*}


\section{Gas-planet interaction}
 
\subsection{Impulsive approx}

planet migration (Lubow ida 10): pg 3
Particle $\Omega(r)$ approach planet $\Omega_p=\Omega(a)$ on a circular orbit: small deflection $r-a\gtrsim R_H$
 


\section{solid accretion}

\subsection{Accrescimento dei planetesimi: collisioni.}
\begin{itemize}
\item $r_{pl}<b<\frac{Gm_{pl}}{v_r^2}=b_0$: incontro ravvicinato ($|U_{max}|>E_{kin}^{\infty}$ l'energia potenziale nel momento di massima interazione sovrasta energia cinetica a infinito).
\item $b>(\frac{m_{pl}}{3\msun{}})\expy{\frac{1}{3}}a=d_{Hill}$: passaggio a grande distanza.
\item $b_0<b<d_{Hill}$: deflessioni di direzione e verso casuali.
\end{itemize}

{Scattering $b>r_{pl}$: variazione velocit\'a relativa}
\begin{equation*}
\delta v_r=\underbrace{\frac{Gm_{pl}}{b^2}}_{\exv{a}}\underbrace{\frac{2b}{v_r}}_{\exv{\tau_{int}}}=\frac{2Gm_{pl}}{bv_r}
\end{equation*}

\begin{equation*}
b_0<b<d_{Hill}:\quad [\TDy{t}{v_r^2}]_{enc}=\int_{r_{pl}}^d2\pi bv_r\frac{\sigma n}{m_{pl}v_r}(\frac{2Gm_{pl}}{bv_r})^2\,db
\end{equation*}

{Urti anelastici: impatti}
\begin{equation*}
[\TDy{t}{v_r^2}]_{imp}=\pi r_{pl}^2(\frac{\sigma n}{m_{pl}v_r})v_r(-v_r)
\end{equation*}

\subsection{Accretion rate: collisions with stick prob 1}
\begin{align}
&\TDy{t}{M}=\underbrace{\rho_p}_{n_mm}v_r\pi R^2[1+(\frac{v_e}{v})^2]\\
&\rho_p\approx\frac{\Sigma}{2a\sin{i}}\approx(\frac{\sqrt{3}}{2})(\frac{\Sigma\Omega}{v})
\end{align}

\chapter{Core Accretion}

\end{document}