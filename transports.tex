\documentclass[main.tex]{subfiles}
 
\begin{document}

\chapter{astro plasma physics}

La presenza dello ione $Z_i$ altera la distribuzione di cariche quindi devo determinare il potenziale $\phi_i$ generato da $Z_i$ e dalla distribuzione di cariche attorno; usando la formula di Boltzmann la densit\'a delle particelle con carica Z risulta
\begin{equation}\label{eq:Bdistroelectricpot}
n_Z=\overline{n}_Z\exp{-\frac{Ze\phi}{kT}}
\end{equation}
con $\overline{n}_Z$ densit\'a numerica della particella di carica $Z$ in assenza di $Z_i$.

L'equazione di Poisson per $\phi$ \'e
\begin{equation}\label{eq:poissonscreened}
\nabla^2\phi=-4\pi e\sum_Z Zn_Z-4\pi e\sum_i Z_i\delta(\vec{r}-\vec{r}_i)
\end{equation}

Nel regime di schermaggio debole, dove l'energia coulombiana \'e molto minore dell'energia termica $e\phi\ll KT$, espando $n_Z$ (\eqref{eq:Bdistroelectricpot}) al prim'ordine quindi in \eqref{eq:poissonscreened} rimane il termine lineare in $\phi$.

Introduco il raggio di Debye:
\begin{equation}
\frac{1}{r_D^2}=\frac{4\pi e^2}{kT}\sum Z^2\overline{n}_Z=\frac{4\pi e^2}{kT}N_A\zeta,\ \zeta=\sum_{i}(Z_i^2+Z_i)\frac{\rho X_i}{A_i}\label{eq:debyeradius}
\end{equation}
quindi dall'equazione \eqref{eq:poissonscreened} ottengo il potenziale generato dallo ione $Z_i$ in $\vec{r}_i$ a distanza $|\vec{r}-\vec{r}_i|=r_i$ come soluzione di:
\begin{equation}\label{eq:poissonspheric}
\frac{r_D^2}{r_i}\TtwoDy{r_i}{(r_i\phi_i)}=\phi_i
\end{equation}
con
\begin{equation}
\phi=\sum_i\phi_i
\end{equation}

\begin{workout}[Raggio di debye e degenerazione elettronica]
Il rapporto fra densit\'a elettronica e quella imperturbata \'e
\begin{equation}
\midfrac{f(\psi-\midfrac{U_e(r)}{KT})}{f(\psi)}
\end{equation}
dove $U_e(r)$ \'e l'energia di interazione di un elettrone
\end{workout}

La soluzione di \eqref{eq:poissonspheric} \'e
\begin{equation}\label{eq:screenedpotential}
\phi_i=\frac{Z_ie}{r_i}\exp{-\midfrac{r}{r_D}}
\end{equation}


\chapter{opacity sources}

\section{Grain opacity in planetary gas accretion}
$\kappa_{gr}=\frac{Q(x)\pi a^2n_{gr}}{\rho}=\frac{3Q(x)f_{d/g}}{4\rho_{gr}a}$
If $a\gg\lambda$ $Q(x)\approx2$ with $x=2\pi a/\lambda$

\section{ISM dust}
%ism-ch3-dust.pdf
%Dust.pdf
%%vanderanker_slides.pdf
%Draine_IPMU_Lectures.pdf

\chapter{Evapopration ??}

Potential
\begin{align*}
&\chi(r)=-\frac{GM_p}{r}+C\\
&\chi'(r)=\chi(r)+\Delta\chi(r)=-G(M_p+M_*)[\frac{\mu}{r}+\frac{1-\mu}{a_p-r}+\frac{((1-\mu)a_p-r)^2}{2a_p^3}]+C\\
&\mu=\frac{M_p}{M_p+M_*}
\end{align*}
Energy needed to escape the planet $\PDy{r}{\chi'}|_{R_h}=0$

\section{Stellar spectrum}

FUV: $\SI{6}{\ev}\leq h\nu\leq\SI{13.6}{\ev}$
EUV: $\SI{13.6}{\ev}\leq h\nu\leq\SI{0.1}{\kilo\ev}$
X-rays: $h\nu\geq\SI{0.1}{\kilo\ev}$

\end{document}