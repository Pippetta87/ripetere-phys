\documentclass[main.tex]{subfiles}

\begin{document}

\part{Fusione}

\chapter{Fenomenologia della fusione}

\subsection{Coulomb barrier}
\begin{align*}
&V=\frac{e^2Z_aZ_X}{R}=1.44\frac{Z_aZ_X}{R(fm)}\\
&B_{aX}=V_{aX}(r=R_a+R_X)=\alpha_{SF}\hbar c\frac{Z_aZ_X}{R_a+R_X}
\end{align*}

\begin{align*}
B_{PP}\approx 0.9 MeV\\
B_{DT}\approx 0.4 MeV
\end{align*}



\section{Fusion between 2 Ne nuclei.}

\subsection{Heavy Ion Collider.}

Accelerating $^{20}Ne$ to $21.2 MeV$ against $^{20}Ne$ target.

\subsection{Thermal energy to overcome Coulomb B.}
For $^{20}Ne$ $\frac{1}{2}*21.2 MeV=\frac{3}{2}KT$ quindi $T=10^{11}K$.

\section{Alcune reazioni di fusione}

\subsection{DD fusion}
\begin{align*}
^2H+^2H\to^4He+\Pphoton\quad(Q=23.8\,MeV)&\intertext{dato che $Q>S_{\Pproton},\,S_{\Pneutron}$ i seguenti processi son pi\'u probabili}\\
^2H+^2H\to^3He+\Pneutron\quad(Q=3.3\,MeV)\\
^2H+^2H\to^3He+\Pproton\quad(Q=4.0\,MeV)
\end{align*}

\section{Energetica della fusione.}

\subsection{Q di fusione: energy release.}

\begin{equation*}
Q=[(m_a+m_X)-(m_b+m_Y)]c^2
\end{equation*}

\subsection{Distribuzione energia cinetica dei prodotti di fusione.}

For most applications of fusion I can neglect the initial motion of reagents

\begin{align*}
&\frac{1}{2}m_bv_b^2+\frac{1}{2}m_Yv_Y^2=Q & \intertext{Energy eq}\\
&m_bv_b=m_Yv_Y & \intertext{Conservazione del momento}
\end{align*}

e ricavo

\begin{equation*}
\frac{T_b}{T_Y}=\frac{m_Y}{m_b}
\end{equation*}

\subsection{Binding energy.}
 Dal grafico della BE/A in funzione di A vedo che la fusione \'e esoenergetica ($Q>0$) fino ad $A=56$ (produco elementi pi\'u legati fino al $^{56}Fe$).


\section{Esempi di Fusion Reaction.}

\subsection{DD reactions.}
$B_{DD}=360KeV$

\begin{align*}
^2H+^2H\to ^4He+\gamma (Q=23.8Mev)\\
^2H+^2H\to ^3He+n (Q=3.3 MeV)\\
^2H+^2H\to ^3H+p (Q=4.0 MeV) \intertext{Lighter products have $75\%$ of Q}
\end{align*}

\subsection{DT reactions}

\begin{align*}
^2H+^3H\to ^4He+n (Q=17.6 Mev)& \intertext{Neutron monoenergetic ($80\%$ of Q)}\\
T_n=\frac{Q}{1+\frac{m_n}{m_{He}}}\approx14.1 MeV &
\end{align*}


\chapter{Reaction Rate di fusione.}

\section{Sezione d'urto di fusione.}

\subsection{Sezione d'urto di Rutherford}

\begin{align*}
\frac{d\sigma}{d\Omega} &=\frac{\text{Events into solid angle $d\Omega$ at $(\theta,\phi)$/(unit time)}}{d\Omega*(\text{Incident particles})*(\text{Target nuclei}/cm^2)}\\
\frac{d\sigma}{d\Omega} &=(\frac{\alpha}{2m{v_{\infty}}^2})^2\frac{1}{\sin^4{\frac{\theta}{2}}}\\
\frac{d\sigma}{d\Omega} &=(\frac{(ze)(Ze)}{4k})^2(\frac{1}{4\pi\epsilon_0})^2\frac{1}{\sin^4{\frac{\theta}{2}}}
\end{align*}

per un sistema di riferimento in cui il centro di massa delle particelle \'e a riposo.

\subsection{Sezione d'urto non risonante}
Particelle interagenti ad energie "termiche" \'e probabile che non siano in zone di risonanza per la sezione d'urto di fusione.

La sezione d'urto \'e proporzionale a:
\begin{enumerate}
\item Fattore $k^{-2}$
\item Partial reaction probability.
Per particelle cariche include un fattore che tiene conto della probabilit\'a di attraversamento della barriera di potenziale 
\begin{align*}
\exp{-2G}=\exp{-\frac{2\pi Z_aZ_Xe^2}{\hbar v}}
\end{align*}
\end{enumerate}

cio\'e la relazione $\sigma_{Fus}\propto\frac{1}{v^2}\exp{-2G}$ include tutta la dipendenza energetica.

Il fattore di proporzionalit\'a coinvolge elementi di matrice nucleare e fattori statistici dipendenti dagli spin.

$\sigma_{FUS}=\sigma(aX\to bY)*P$:

P \'e la probabilit\'a di attraversamento della barriera Coulombiana.

$\sigma(aX\to bY)$ dipende da elementi di matrice nucleare e fattori statistici relativi agli spin delle particelle.

Per includere la debole dipendenza da E per fusione non risonante scrivo:

$\sigma_{FUS}=\frac{S(E)}{E}\exp{-2G}$:

$S(E)$ \'e chiamato fattore astrofisico.

La dipendenza $\propto\frac{1}{E}$ \'e un fattore geometrico.

\subsection{Probabilit\'a attraversamento barriera Coulombiana}
La probabilit\'a di attraversare una barriera infinitesima $[r,r+\,dr]$ 
\begin{align*}
dP=\exp{-2\,dr\sqrt{\frac{2m}{\hbar^2}}\sqrt{V(r)-E}}&\intertext{che per barriera finita}\\
P=\exp{-2G}
\end{align*}

Energia di Gamow: $E_G$.

\begin{align*}
P\propto\exp{-\sqrt{\frac{E_G}{E}}}&\intertext{Definita l'energia}\\
E_G=2mc^2(\pi \alpha_{SF}Z_aZ_X)^2
\end{align*}

\subsection{"Gamow Factor": alpha decay vs fusion}

Avendo in mente la probabilit\'a di attraversamento di una barriera Coulombiana calcolata nella teoria del alpha decay.

\begin{align*}
G=\sqrt{\frac{2m}{\hbar^2}}\int_a^b\sqrt{V(r)-E}\,dr & \intertext{con $x=\frac{Q}{B}$}\\
&\intertext{nel caso della fusione sostutuisco Q con E=center of mass energy of reacting particles}
E\approx KeV\ll B&\intertext{approssimo}\\
G\approx\frac{e^2}{4\pi\epsilon_0}\frac{\pi Z_aZ_X}{\hbar v}=\pi\alpha_{SF}\frac{Z_AZ_X}{v/c}
\end{align*}

Riscrivo G: per praticit\'a raggruppo le costanti del fattore di Gamow in $\beta$.

\begin{equation*}
G=\beta\frac{1}{\sqrt{E}}=\frac{\pi}{2}\alpha_{SF}\sqrt{2mc^2}Z_aZ_X\frac{1}{\sqrt{E}}=\pi\alpha_{SF}\frac{Z_AZ_X}{v/c}
\end{equation*}

\section{Picco di Gamow}

\subsection{Distro di MB}
\begin{align*}
F_{MB}(v)=4\pi v^2(\frac{m}{2\pi k_B T})^{\frac{3}{2}}\exp{-\frac{mv^2}{2K_BT}}
\end{align*}

\subsection{Reaction Rate: costante di decadimento}
\begin{equation*}
R(aX\to bY)=\frac{\# \text{reazioni}}{\text{Tempo}*\text{Volume}}=\sigma(aX\to bY)n_X\Phi_a
\end{equation*}

definito il flusso delle particelle a 
\begin{equation*}
\Phi_a=\frac{N_a}{\text{Superficie}*\text{Tempo}}=\frac{n_aV}{S*T}=n_av
\end{equation*}
quind
\begin{align*}
R&=n_an_X\exv{\sigma(v_{\text{rel}}) v_{\text{rel}}}\\
&=n_an_X\lambda&\intertext{definito il reaction rate per coppia di particelle ($n_a=1/cm^3$)}\\
\lambda=\exv{\sigma_Fv}\\
R=\frac{1}{2}n_X^2\exv{\sigma v}& \intertext{per particelle identiche.}
\end{align*}

La densit\'a di potenza prodotta in una reazione binaria, detto $W_{aX}$ l'energia liberata in una singola reazione, \'e
\begin{equation*}
P_{aX}=n_an_X\exv{\sigma v}W_{aX}
\end{equation*}

\subsection{Massimo di $F_{MB}\sigma v$}

\begin{align*}
\exv{\sigma v}\propto\intzi\frac{1}{v}\exp{-2G}\exp{-\frac{mv^2}{2kT}}v^2\,dv\\
\exv{\sigma v}\propto\intzi\exp{-2G}\exp{-\frac{E}{kT}}\,dE
\end{align*}

\begin{align*}
\exv{\sigma_Fv}&=\intzi\sigma_FvF^{MB}(v)\,dv\\
&=\sqrt{\frac{8}{\pi m}}(K_BT)^{-\frac{3}{2}}\intzi S(E)\exp{-\frac{E}{KT}-\frac{\beta}{\sqrt{E}}}\\
&\approx\sqrt{\frac{8}{\pi m}}(K_BT)^{-\frac{3}{2}}S_0\intzi \exp{-\frac{E}{KT}-\frac{\beta}{\sqrt{E}}}&\intertext{tutta ci\'o che non sappiamo sono i fattori astrofisici estrapolati a basse energia.}
\end{align*}
\index{Fattore astrofisico: energia zero.}

L'integrando \'e fortemente piccato quindi le particelle che danno maggiormente luogo a fusione (a parit\'a di condizioni) son quelle con picco pi\'u alto
\begin{equation*}
\frac{d}{dE}(\frac{E}{KT}+bE^{-\frac{1}{2}})_{E=E_0}=\frac{1}{KT}-\frac{1}{2}b(\frac{1}{E_0})^{\frac{3}{2}}
\end{equation*}
ovvero

\begin{align*}
&E_0=(\frac{bKT}{2})^{\frac{2}{3}}&\intertext{con}\\
A=\frac{A_1A_2}{A_1+A_2}=\frac{m}{m(u)}\\
&E_0=1.22(Z_a^2Z_X^2AT_6^2)^{\frac{1}{3}} (keV)&\intertext{Most effective energy for thermonuclear reactions}
\end{align*}

\index{Energia del picco di Gamow}

\subsection{Variazione di concentrazione dei reagenti}

\begin{align*}
&R=n_an_X\exv{\sigma v}&\intertext{Reaction rate medio}\\
&\lambda_a(X)=\frac{1}{\tau_a(X)}=\frac{\text{Probabilit\'a di fusione a-X}}{\text{Tempo}*\text{Nucleo X}}
\end{align*}

quindi
\begin{align*}
\frac{\partial n_X(t)}{\partial t}|_a&=-\lambda_a(X)n_X(t)=-\frac{n_X(t)}{\tau_a(X)}\\
&=-R_{aX}=-n_an_X\exv{\sigma_{aX}v}&\intertext{la vita media per la reazione $a+X$ \'e}\\
\tau_a(X)=\frac{1}{n_a\exv{\sigma_{aX}v}}
\end{align*}

\section{Risonanze nella fusione}

\subsection{Low energy capture reaction: compound nucleus}

La sezione d'urto di fusione di alcune reazioni ha dei picchi di risonanza:
\begin{equation*}
^1H+^{12}C\to^{13}N+\gamma\quad E\approx0.5 MeV
\end{equation*}

\begin{equation*}
\sigma_{ris}=\frac{\pi}{k^2}\frac{2J+1}{(2J_a+1)(2J_X+1)}\frac{\Gamma_{aX}\Gamma_{bY}}{(E-E_R)^2+\frac{\Gamma^2}{4}}
\end{equation*}

\part{Meccanismo di fusione nel stelle}
  
\chapter{Reazioni di base}

\section{PP cycle}

Nell'interno stellare l'idrogeno \'e ionizzato quindi nel caso di emissione di positroni nel Q-valore \'e compreso un contributo per l'annichilamento \Pelectron\APelectron.

\subsection{Bottle-neck}

\begin{align*}
&^1H+^1H\to^2H+\APelectron+\Pnue (Q=1.44 MeV)&\intertext{Il neutrino ha spettro continuo con endpoint}\\
&E(\nu)=0.42 MeV
\end{align*}

Reazione lenta:

\begin{align*}
&\sigma\approx 10^{-33}b (KeV)\\
&\approx 10^{-23} b (MeV)\\
&R=\frac{1}{2}n_P^2\exv{\sigma v}\approx5*10^{-18} \text{reazioni}/\text{P}/s\\
&\rhosunc=125gr/cm^3 \quad (7.5*10^{25}P/cm^3)\\
&\tsunc 15*10^6K\Rightarrow\exv{T_P}\approx1 KeV&
\intertext{per le regioni centrali del sole.}
\end{align*}

\subsection{Deuteron cooking up to Helium}

Il deutone viene trasformato rapidamente in isotopo di elio
\begin{equation*}
^2H+^1H\to ^3He+\gamma \quad (Q=5.49 MeV)
\end{equation*}

\subsection{Produzione di He4: ciclo PP1 (Sun:$69\%$).}
\begin{equation*}
^3He+^3He\to^4He+2^1H+\gamma\quad(Q=12.86 MeV)
\end{equation*}

\subsection{Bilancio energetico PP}
L'energia dei neutrini che escono dal core non scalda la fotosfera.
\begin{align*}
Q=26.7 MeV &\intertext{Atomi neutri e annichilamento \Pelectron\APelectron.}
\end{align*}

\subsection{Tempi medi di reazione nella catena PP.}
Per le condizioni presenti nel sole:

\begin{tabular}{c|c|}
\hline
Reazione & $t_r$ \\
\hline
$^1H+^1H\to^2H+\APelectron+\Pnue$ & $7*10^9\,yr$\\
$^2H+^1H\to ^3He+\gamma$ & $4 s$\\
$^3He+^3He\to^4He+2^1H$ & $4*10^5\,yr$\\
$^{12}C+^1H\to ^{13}N+\gamma$ & $10^6\, yr$\\
$^{13}N\to^{13}C+\APelectron+\Pnue$ & $10\,min$\\
$^{13}C+^1H\to ^{14}N+\gamma$ & $2*10^5\, yr$\\
$^{14}N+^1H\to ^{15}O+\gamma$ & $<3*10^7\, yr$\\
$^{15}O\to^{15}N+\APelectron+\Pnue$ & $2\,min$\\
$^{15}N+^1H\to ^{12}C+^4He$ & $10^4\, yr$\\
\hline
\end{tabular}

\section{Altre diramazioni della catena \Pproton\Pproton dopo produzione He3.}

\subsection{Ciclo HeP (Sun: $0.0001\%$).}
\begin{align*}
^3He+^1H\to^4He+\APelectron+\Pnue \quad (Q=19.28 MeV)\\
E_{\nu}^{max}=Q-m_ec^2=18.77 MeV
\end{align*}

\subsection{Produzione Be7 (Sun: $31\%$).}
\begin{equation*}
^3He+^4He\to^7_4Be+\gamma
\end{equation*}
da cui seguono 2 diramazioni:

\subsection{Ciclo PP2 (Sun: $99.7\%$).}
\begin{align*}
^7_4Be+\Pelectron\to^7_3Li+\Pnue&\intertext{CE: decadimento a 2 corpi: neutrino monoenergetico}\\
E(\nu)\approx0.862 MeV\\
^7_3Li+^1H\to2^4He
\end{align*}

\subsection{Ciclo PP3 (Sun: $0.3\%$).}
\begin{align*}
^7_4Be+^1H\to^8_5B+\gamma\\
^8_5B\to^8_4Be+\APelectron+\Pnue&\intertext{Spettro energetico del neutrino continuo con endpoint}\\
E(\nu)=14 MeV\\
^8_4Be\to2^4He
\end{align*}

\subsection{Energia irradiata in neutrini}
\begin{align*}
Q_{eff}&=Q-\exv{E_{\nu}}(MeV)\quad&\text{Perdita in neutrini}&\intertext{PP1}\\
&=26.2 &2\%&\intertext{PP2}\\
&=25.66  & 4\%&\intertext{PP3}\\
&=19.17  & 28\%
\end{align*}


\section{Ciclo CNO}
Procede con maggiore velocit\'a perch\'e non \'e presente il bottleneck $H+H\to D$ ma la barriera coulombiana per la fusione dei nuclei \'e 6-7 volte pi\'u elevata: efficiente ad alte temperature.


\begin{align*}
^{12}_6C+^1H\to^{13}_7N+\gamma\\
^{13}N\to ^{13}_6C+\APelectron+\Pnu\\
^{13}C+^1H\to ^{14}_7N+\gamma\\
^{14}_7N+^1H\to ^{15}_8O+\gamma\\
^{15}_8O\to ^{15}_7N+\APelectron+\gamma\\
^{15}_7N+^1H\to ^{12}_6C+^4_2He
\end{align*}

Processo efficace:

$4^1H\to ^4He+2\APelectron+2\Pnue$ uguale alla catena PP, stesso Q-valore.

\clearpage

\chapter{Evoluzione stellare}

\section{Modello standard del sole}

\subsection{Il sole.}

\begin{align*}
&\msun=2*10^{33}gr\\
&\text{Age}=4.55*10^9\,yr\\
&\rsun=7*10^5 Km\\
&\tsunc=1.57*10^7 K\\
&\tsuns=6*10^3 K\\
&\rhosunc=130gr/cm^3\\
&\lsun=3.8*10^{26}W=3.8*10^{33}erg/s\\
&\Phi_{\nu}(\Pproton\Pproton)=6.0*10^{10}cm^{-2}s^{-1}\\
&\Phi_{\nu}(^8B\to^8Be^*+\APelectron+\Pnue)=6.0*10^6cm^{-2}s^{-1}\\
&X=74\%,\,Y=23,\,Z=3&\intertext{$\uparrow$ Attuale}\\
&X=75\%,\,Y=22,\,Z=3&\intertext{$\uparrow$ Composizione iniziale}\\
\end{align*}

\subsection{Equilibrio idrostatico.}
\begin{align*}
dS\,(P(r)-P(r+\,dr)-F_g(r))=0\\
F_g(r)=G\frac{m(r)\rho(r)\,dV}{r^2}&\intertext{se ho simmetria sferica}\\
\frac{dm}{dr}=4\pi\rho(r)r^2
\end{align*}

Posso trascurare la BE gravitazionale?
\begin{align*}
B_{\odot}=N_Pm_p-\msun=-E_G=\frac{3}{5G\frac{\msun^2}{\rsun}}\approx10^{-6}\msun c^2
\end{align*}

\subsection{Equazione di stato.}

\begin{align*}
&P=P(\rho,T,X,Y,Z)=P_{Gas}+P_{Rad}&\intertext{Gas ideale}\\
&P_{Gas}=K\rho T\\
&P_{Rad}=\frac{1}{3}aT^4
\end{align*}

\subsection{Produzione di energia}
\begin{align*}
\epsilon=\frac{\text{Energia prodotta}}{\text{Massa}*\text{Tempo}}&\intertext{L'energia prodotta per fusione}\\
\epsilon=\epsilon_F(\exv{\sigma v},\rho,T,X,Y,Z)&\intertext{\'e legata alla luminosita tramite}\\
\frac{d\lu}{dr}=\epsilon_F(r)4\pi r^2
\end{align*}

\subsection{Reaction Rate}
La radiazione che raggiunge la terra ha un flusso di $1.4*10^3 W/m^2$ quindi la potenza irraggiata \'e $4*10^{26} W$: dato che ogni reazione di fusione libera 25 MeV sono necessarie $10^{38} \text{Fusioni}/sec$ (4 $^1H$ per fusione).

\begin{equation*}
N_{Reaz}=\frac{\exv{R_{PP}}}{n_P}N_p(core)
\end{equation*}

\subsection{Flusso neutrini}
\begin{equation*}
\Phi_{\nu}=\frac{\dot{N_{\nu}}}{4\pi d^2}=7*10^{10}\frac{\Pnu}{s*cm^2}
\end{equation*}

\section{Fasi finali dell'evoluzione stellare.}

\subsection{Teorema del viriale}
Relazione tra l'energia cinetica media e l'energia potenziale di un sistema in equlibrio stabile o quasi-stabile.

\begin{align*}
&G=\sum_i\scap{p_i}{r_i}=\frac{1}{2}(\sum_im_ir_i^2)^2=\frac{1}{2}\frac{dI}{dt}&\intertext{I momento d'inerzia}\\
&\frac{dG}{dt}=\sum_i\scap{\dot{r_i}}{p_i}+\sum_i\scap{r_i}{\dot{p_i}}=2T+\sum_i\scap{r_i}{f_i}&\intertext{l'ultimo termi del membro di destra \'e il  viriale di Clausius}\\
&\intertext{Assumendo che la forza tra le particelle segua una legge di potenza della distanza con esponente n(del potenziale) e sia derivabile da un potenziale}\\
&\frac{dG}{dt}=2T-n\mathcal{U}&\intertext{dove U \'e l'energia potenziale totale del sistema}\\
&\frac{dG}{dt}=\frac{1}{2}\frac{d^2I}{dt^2}=2T+\Omega&\intertext{per il potenziale gravitazionale n=-1}\\
&2\overline{T}+\overline{\Omega}=0&\intertext{se il sistema \'e periodico la media del lato sinistro su un periodo dell'equazione sopra \'e nullo}
\end{align*}

\subsection{Fasi di contrazioni ed espansioni successive.}

Terminato l'idrogeno il core di elio si contrae non essendoci pi\'u fonti di energia fino a che si raggiungono T sufficienti per la fusione di dell'elio.

Detti m massa del core e r suo raggio
\begin{align*}
E_G=-G\frac{m^2}{r}&\intertext{quindi in fase di contrazione}\\
\Delta E_G=\frac{Gm}{r^2}\Delta r<0&\intertext{da cui risulta aumento di energia cinetica}\\
\Delta E=-\frac{1}{2}\Delta U_G>0
\end{align*}

\subsection{fase di fusione dell'elio ($T\approx10^8K$).}

Fusione di elio-4 in carbonio-12
\begin{align*}
^4He+^4He\to ^8_4Be\quad(\tau\approx10^{-16}s)\\
^8Be+^4He\to^{12}C+\gamma&\intertext{al crescere di T}\\
^{12}C+^4He\to ^{16}O+\gamma\\
^{16}O+^4He\to^{20}Ne+\gamma
\end{align*}

\subsection{Fase di fusione del carbonio-12.}
\begin{align*}
^{12}C+^{12}C&\to ^{16}O+2^4He\\
&\to ^{20}Ne+^4He\\
&\to ^{23}Na+\Pproton\\
&\to^{24}Mg+\gamma
\end{align*}

\subsection{Fase fusione Ossigeno.}
\begin{align*}
^{16}O+^{16}O&\to ^{28}_{14}Si+^4He\\
&\to^{31}_{15}P+\Pproton\\
&\to ^{32}S+\gamma
\end{align*}

e cos\'i via fino al $^{56}Fe$ per le stelle pi\'u massicce.

\subsection{Destino finale della stella}

\begin{itemize}
\item $M<8\msun$: Nane bianche ($R\approx R_T$).
La pressione di degenerazione degli elettroni blocca il collasso.
\item $8\msun<M<25\msun$: stella di neutroni.
La nucleosintesi raggiunge il $^{56}Fe$  poi hanno luogo fenomeni di cattura elettronica (neutronizzazione della materia): sintesi elementi pi\'u pesan
\item $M>25\msun$: buco nero.
\end{itemize}


\section{Nane bianche}

\subsection{Grandezze caratteristiche}

\begin{itemize}
\item $M\approx\msun$
\item $\rho_c\approx10^6gr/cm^3\approx10^4\rhosunc$.
\begin{align*}
n_e=\exv{Z}n_I\\
\rho&=\exv{m_I}n_I+m_en_e\\
&\approx\exv{M_{atom}}n_I=\exv{A}m_Hn_I
\end{align*}

\item Raggio di Schwarzschild.

\begin{equation*}
\frac{R_G}{R}\approx100(\frac{R_G}{R})_{\odot}\approx4*10^{-6}
\end{equation*}
ho definito il raggio di Schwarzchild come $R_G=\frac{2GM}{c^2}$.
\item Velocit\'a di fuga.

\begin{equation*}
v_f=\sqrt{\frac{2GM}{R}}\approx0.02 c
\end{equation*}

\item Pressione.

\begin{equation*}
P=P_I+P_{el}\approx P_{el}
\end{equation*}
\end{itemize}

\subsection{How far inward before degeneracy set in?}
Condizione necsuff perch\'e si abbia degenerazione \'e
\begin{align*}
\frac{4\pi^2}{(\theta mc^2)^2}\frac{x\sqrt{x^2+1}}{f(x)}\gg1&\intertext{ho definito}\\
x=\frac{h}{mc}(\frac{3n}{8\pi})^{\frac{1}{3}}\quad \theta=\frac{1}{KT}
\end{align*}
Per gli strati esterni vale $P\propto\rho T$ ($\approx0.23 \%$ della massa) 


\subsection{Equazione di stato gas perfetto}

\begin{align*}
n=\frac{N}{V}\\
\rho=\frac{M}{V}\\
m_0=\frac{M}{N}&\intertext{\'e il peso molecolare medio}\\
\end{align*}
ho l'equazione di stato
\begin{align*}
PV=nRT=NKT&\intertext{Riscrivo l'equazione di stato esprimendo la pressione in funzione della densit\'a, temperatura e composizione}\\
P=(\frac{N}{V})KT=nKT=\frac{\rho}{m_0}KT\\
P=\frac{1}{\mu}\frac{K_B}{m_P}\rho T&\intertext{definisco il peso molecolare in termini di proton mass}\\
\frac{1}{\mu}=2X+\frac{3}{4}Y+\frac{Z}{2}
\end{align*}
\index{Peso molecolare: particelle libere. ???}

Ricavo la relazione densit\'a di energia cinetica-pressione.

\begin{align*}
\epsilon=\frac{E}{V}=\frac{1}{V}\frac{3}{2}NK_BT&\intertext{Relazione tra pressione e concentrazione degli elettroni.}\\
P=\frac{2}{3}\epsilon
\end{align*}

\subsection{Pressione di radiazione}
For radiative equilibrium
\begin{equation*}
F_{RAD}=-\frac{d}{dr}(\frac{a}{3}T^4)=-\frac{dP_R}{dr}
\end{equation*}

\subsection{Gas di Fermi non relativistico}

\begin{itemize}

\item Degenerazione.

\begin{equation*}
n_e\,d^3x\,d^3p\leq2\frac{d^3x\,d^3p}{h^3}
\end{equation*}

\item Densit\'a di stati.


\begin{align*}
dN=\frac{1}{8}4\pi n^2dn&\intertext{da cui ricavo la densit\'a imponendo le condizioni periodiche al bordo del volume $V=L^3$}\\
g(k)=\frac{dN(k)}{d^3k}=\frac{dN(k)}{4\pi k^2dk}=\frac{\nu V}{(2\pi)^3}
\end{align*}

\item Energia gas di Fermi completamente degenere: $T=0$.
\begin{equation*}
N=\int g(k)F(k)d^3k=\frac{\nu V}{(2\pi)^3}\int_0^{K_F}4\pi k^2\,dk=\frac{\nu V}{2\pi^2}\frac{k_F^3}{3}
\end{equation*}
da cui ricavo la densit\'a
\begin{equation*}
n=\frac{\nu}{6\pi^2}k_F^3
\end{equation*}

e adesso introduco la densit\'a di energia e ricavo la sua dipendenza da n
\begin{align*}
\epsilon=\frac{\hbar^2k^2}{2m_0}\quad(\epsilon_F=\frac{\hbar^2k_F^2}{2m_0})\\
E=\int\epsilon(k)F(k)g(k)d^3k=\frac{\nu V}{2\pi^2}\frac{\hbar^2}{2m_0}\frac{k_F^5}{5}\\
\frac{E}{N}=\frac{3}{5}\epsilon_F\\
\epsilon=\frac{E}{V}=\frac{E}{N}\frac{N}{V}=\frac{3}{5}\epsilon_Fn\propto n^{\frac{5}{3}}
\end{align*}
e per la pressione di un gas di Fermi di elettroni completamente degenere
\begin{align*}
P&=-\frac{\partial E}{\partial V}|_{S,N}=n^2\frac{\partial(\frac{E}{N})}{\partial n}|_{S,N}\propto n^{\frac{5}{3}}\\
P&=K_{NR}\rho^{\frac{5}{3}}
\end{align*}

\end{itemize}



\subsection{Gas di Fermi relativistico}
\begin{itemize}

\item Energia.

\begin{equation*}
\epsilon(k)=\sqrt{(\hbar ck)^2+(m_0c^2)^2}
\end{equation*}

\item Parametri adimensionali.

\begin{align*}
y=\frac{\hbar}{m_0c}k=\lambdabar k\\
\lambdabar k_F=x
\end{align*}

\item densit\'a di particelle.

\begin{equation*}
n=\frac{8\pi c^3m^3}{3h^3}x^3=5.87*10^{29}x^3
\end{equation*}

\item Energia cinetica.

\begin{equation*}
P\propto n^{\frac{4}{3}}
\end{equation*}

\end{itemize}

\subsection{Relazione elettroni/Ioni}
\begin{equation*}
N_E=\frac{1}{\mu_E}\frac{\rho}{m_P}\quad\frac{1}{\mu_E}=X+\frac{1}{2}Y+\frac{1}{2}Z\approx\frac{1}{2}(1+X)
\end{equation*}

Condizioni per la degenerazione degli atomi: nuclei have on the average the same energy as electron, il momento \'e maggiore di un fattore radice quadrata del rapporto fra le masse per grado di libert\'a e quindi sono distribuiti nello spazio delle fasi su un volume $(2000)^{\frac{3}{2}}$ maggiore.
Per gli atomi possiamo usare
\begin{equation*}
P_A=\frac{1}{\mu_A}\frac{K}{m_P}\rho T\quad\frac{1}{\mu_A}=X+\frac{1}{4}Y
\end{equation*}

\subsection{Boundaries between D/ND and in latter R/NR}

Electron pressure for complete degeneracy
\begin{equation*}
P_E=\int_{|p|\leq p_0}\,d^3p\,p_xv_x\frac{2}{h^3}
\end{equation*}
e utilizzando l'opportuna relazione (R/NR) tra v e p trove la pressione nei due casi per gas di elettroni completamente degenere

\begin{itemize}
\item Relativistico: $v=\frac{p}{m}$.
\begin{equation*}
P_E=\frac{2\pi c}{3h^3}p_0^4=K_2(\frac{\rho}{\mu_E})^{\frac{4}{3}}
\end{equation*}

\item Non relativistico $v_x=c\frac{p_x}{|p|}$.
\begin{equation*}
P_E=\frac{8\pi}{15mh^3}p_0^5=K_1(\frac{\rho}{\mu_E})^{\frac{5}{3}}
\end{equation*}

\end{itemize}

\begin{itemize}
\item Confine tra zona degenere e non degenere.

\begin{align*}
\frac{1}{\mu_E}\frac{K}{m_P}\rho T=K_1(\frac{\rho}{\mu_E})^{\frac{5}{3}}\\
\frac{1}{\mu_E}\rho=(\frac{KT}{m_PK_1})^{\frac{3}{2}}
\end{align*}
\item Confine tra zona degenere relativistica e non.

\begin{equation*}
\frac{1}{\mu_E}\rho=(\frac{K_2}{K_1})^3=1.916*10^6
\end{equation*}

\end{itemize}

Il sole \'e descrivibile con equazioni non degeneri le nane bianche con equazioni degeneri.

\subsection{Mass/radius relation.}
L'equazioni di stato \'e indipendente da T
\begin{align*}
P=\frac{\pi m^4c^5}{3h^3}f(x)\\
\rho=\mu_E\frac{8\pi m_Pm^3c^3}{3h^3}x^3&\intertext{con}\\
x=\frac{p_0}{mc}\quad \mu_E=\frac{2}{1+X}
\end{align*}
Posso separare il problema idrostatico da quello termodinamico.

Equilibrio idrostatico:
\begin{align*}
\frac{dP}{dr}=-\rho\frac{GM_r}{r^2}\\
\frac{dM_r}{dr}=4\pi r^2\rho
\end{align*}
da integrare numericamente(Chandrasekhar).

Qualitativamente
\begin{itemize}
\item The larger the mass of a white dwarf the smaller the radius.
\item Per nane bianche con circa massa solare il raggio \'e centinaia di volte pi\'u piccolo del sole.
\end{itemize}

Limite superiore per la massa di una stella completamente degenere.

Quantit\'a che influenzano l'equilibrio idrostatico:
\begin{align*}
\rho\propto\frac{M}{R^3}&\intertext{questa sopra \'e la densit\'a, questa sotto l'attrazione gravitazionale}\\
\rho\frac{GM_r}{r^2}\propto\frac{M^2}{R^5}
\end{align*}

Per la pressione:
\begin{align*}
P\propto\rho^{\frac{5}{3}}\propto\frac{M^{\frac{5}{3}}}{R^5}&\intertext{per stelle meno massive ho la pressione NR (sopra), per stelle massive uso la formula R (sotto)}\\
P\propto\rho^{\frac{4}{3}}\propto\frac{M^{\frac{4}{3}}}{R^4}
\end{align*}

il cui gradiente genera una forza di pressione
\begin{align*}
F_P&\propto\frac{M^{\frac{5}{3}}}{R^6}\quad\text{NR-D}\\
&\frac{M^{\frac{4}{3}}}{R^5}\quad\text{R-D}
\end{align*}

Osservazioni:
\begin{itemize}
\item In the relativistic case gravitational force and pressure force depend on the same power of radius.
\item In R-D the 2 forces depend on different power of mass.
\end{itemize}

\subsection{Politrope}
Chiamo la funzione di sotto politropa di ordine n
\begin{equation*}
P=K\rho^{\gamma}=K\rho^{\frac{n+1}{n}}
\end{equation*}

\begin{itemize}
\item Caso NR.
\begin{equation*}
P=K_{NR}\rho^{\frac{5}{3}}\quad N=\frac{3}{2}
\end{equation*}

\item Caso R.
\begin{equation*}
P=K_{UR}\rho^{\frac{4}{3}}\quad n=3
\end{equation*}
\end{itemize}


Equazione di Lane-Emden:
Considero l'equilibrio idrostatico
\begin{align*}
\frac{dP}{dr}&=-G\frac{\rho(r)m(r)}{r^2}\\
\frac{dm}{dr}&=4\pi r^2\rho(r)&\intertext{con le condizioni al contorno:}\\
\rho(r=0)&=\rho_c\quad \rho=\frac{m_p}{Y_e}n_e=\mu_em_pn_e\quad\frac{1}{Y_e}=\frac{\exv{A}}{\exv{Z}}\\
\frac{d\rho}{dr}|_0&=0
\end{align*}
\index{Peso molecolare medio elettroni}

Riscrivo l'equazione dell'equilibrio idrostatico
\begin{equation*}
\frac{d}{dr}(\frac{r^2}{\rho(r)}\frac{dP}{dr})=-G\frac{dm(r)}{dr}=-G4\pi r^2\rho(r)
\end{equation*}
e ipotizzo 
\begin{align*}
\rho(\xi)=\rho_c\theta^n(\xi)&\intertext{e ho definito a con}\\
r=a\xi\quad a^2=\frac{(n+1)K}{4\pi G}\rho_c^{\frac{1-n}{n}}\\
P=K\rho^{\frac{n+1}{n}}=P_c\theta^{n+1}(\xi)
\end{align*}
per trovare la pressione devo risolvere l'equazione di Lane-Emden
\begin{align*}
\frac{1}{\xi^2}\frac{d}{d\xi}(\xi^2\frac{d\theta}{d\xi})=-\theta^n(\xi)&\intertext{con le condizioni al contorno}\\
\theta(\xi=0)=1\quad\frac{d\theta}{d\xi}|_{\xi=0}=0
\end{align*}

Raggio stellare: $\rho(r=R)=0$, primo zero della $\theta$.
\begin{equation*}
R=a\xi_1=B_n\rho_c^{\frac{1-n}{2n}}\quad\theta^n(\xi_1)=0
\end{equation*}

Massa stellare: $M=m(R)$.
\begin{equation*}
M=m(R)=m(a\xi_1)=\int_0^R\rho(r)4\pi r^2\,dr=A_n\rho_c^{\frac{3-n}{2n}}
\end{equation*}

Elimino la densit\'a centrale e ottendo una relazione tra massa e raggio
\begin{equation*}
M(n,R)=\frac{A_n}{B_n^{\frac{3-n}{1-n}}}R^{\frac{3-n}{1-n}}
\end{equation*}

\begin{itemize}
\item Caso NR: $n=\frac{3}{2}$.
\begin{equation*}
M_{\frac{3}{2}}(R)\propto R^{-3}
\end{equation*}
\item Caso R: $n=3$.
\begin{equation*}
M_3(R)=const
\end{equation*}
\end{itemize}

Trovo che esiste una massa limite per raggio tendente a zero e densit\'a infinita:
\begin{equation*}
M_{CSK}=M_3(R)=\frac{5.8\msun}{\mu_e^2}
\end{equation*}

\subsection{Interazione Ioni-elettroni}
Il contributo Coulombiano alla pressione diminuisce la massa limite:
\begin{equation*}
P_C\propto-e^2\exv{Z}^{\frac{2}{3}}n_e^{\frac{4}{3}}<0
\end{equation*}

%http://dujs.dartmouth.edu/wp-content/uploads/2008/04/09-whitedwarf.pdf
%http://upcommons.upc.edu/e-prints/bitstream/2117/18798/1/303_bravo.pdf
%http://www.astro.uvic.ca/~fherwig/TEACHING/ASTR501/StudentPresentations/EOS%20ASTR401%20MichaelP.pdf
%http://w.astro.berkeley.edu/~gmarcy/astro160/papers/physics_of_white_dwarfs.pdf
%

\end{document}